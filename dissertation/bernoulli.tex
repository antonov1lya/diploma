\subsection{Трехмерное распределение Бернулли}
В работах \cite{Dai2013,Teugels1990} трехмерное распределение Бернулли вводится следующим образом.
\begin{definition}
    Случайный вектор $(X,Y,Z)^T$ имеет трехмерное распределение Бернулли,
    если множество его возможных значений:
    $$
        \begin{pmatrix}
            0 \\
            0 \\
            0
        \end{pmatrix},
        \begin{pmatrix}
            0 \\
            0 \\
            1
        \end{pmatrix},
        \begin{pmatrix}
            0 \\
            1 \\
            0
        \end{pmatrix}, \ldots, \begin{pmatrix}
            1 \\
            1 \\
            1
        \end{pmatrix}
    $$ и заданы вероятности:
    $$P(X=x,Y=y,Z=z)=p_{xyz} \geq 0,  \sum_{x=0}^1 \sum_{y=0}^1 \sum_{z=0}^1 p_{xyz} =1$$
\end{definition}


Покажем, что трехмерное распределение Бернулли принадлежит к многопараметрическому экспоненциальному семейству \cite{Lehmann1986}.
\begin{lemma}\label{factorization}
    Трехмерное распределение Бернулли принадлежит к многопараметрическому экспоненциальному семейству, то есть:
    $P(X=x,Y=y,Z=z)= C(\theta) \exp\left\{ \sum_{i=1}^7 \theta_i T_i \right\}$, 
    где
    $$
    P(X=x,Y=y,Z=z)= p_{000}
        \exp \Biggl\{  xyz \ln  \left(\dfrac{p_{001}p_{111}p_{010}p_{100}}{p_{011}p_{101}p_{000}p_{110}}\right) +$$
    $$ +
        x \ln\left(\dfrac{p_{100}}{p_{000}}\right) +  y \ln\left(\dfrac{p_{010}}{p_{000}}\right) +
        z \ln\left(\dfrac{p_{001}}{p_{000}}\right) +
    $$
    $$
        + xy \ln \left(\dfrac{p_{000}p_{110}}{p_{010}p_{100}}\right) +
        xz \ln \left(\dfrac{p_{000}p_{101}}{p_{001}p_{100}}\right) +
        yz \ln \left(\dfrac{p_{000}p_{011}}{p_{001}p_{010}}\right) \Biggr\}
    $$
\end{lemma}

\begin{proof}
    $$
    P(X=x,Y=y,Z=z) = p_{000}^{(1-x)(1-y)(1-z)} \ldots p_{111}^{x y z} =
    $$
    $$
        =\exp \Biggl\{ (1-x)(1-y)(1-z) \ln p_{000} +
        (1-x)(1-y)z \ln p_{001}+
    $$
    $$
        + (1-x)y(1-z) \ln p_{010} + (1-x)y z \ln p_{011} +  x(1-y)(1-z) \ln p_{100} +
    $$
    $$
        +   x(1-y) z \ln p_{101}
        +   x y (1-z) \ln p_{110} +   x y z \ln p_{111} \Biggr\} =
    $$
    $$
        =\exp \Biggl\{   ( 1 - y -  x +  x y
        -  z +  y z +  x z -  x y z ) \ln p_{000} +
    $$
    $$
        +    (z -  y z -  x z +  x y z) \ln p_{001}  +
          (y -  y z -  x y +  x y z)  \ln p_{010} +
    $$
    $$
        +    (y z -  x y z ) \ln p_{011} +
           (x -  x z -  x y +  x y z ) \ln p_{100} +
    $$
    $$
        +   (x z -  x y z ) \ln p_{101} +   (x y -  x y z) \ln p_{110} +
          x y z \ln p_{111} \Biggr\}=
    $$
    $$
        = p_{000}
        \exp \Biggl\{  xyz \ln  \left(\dfrac{p_{001}p_{111}p_{010}p_{100}}{p_{011}p_{101}p_{000}p_{110}}\right) +$$
    $$ +
        x \ln\left(\dfrac{p_{100}}{p_{000}}\right) +  y \ln\left(\dfrac{p_{010}}{p_{000}}\right) +
        z \ln\left(\dfrac{p_{001}}{p_{000}}\right) +
    $$
    $$
        + xy \ln \left(\dfrac{p_{000}p_{110}}{p_{010}p_{100}}\right) +
        xz \ln \left(\dfrac{p_{000}p_{101}}{p_{001}p_{100}}\right) +
        yz \ln \left(\dfrac{p_{000}p_{011}}{p_{001}p_{010}}\right) \Biggr\}
    $$
\end{proof}

\subsection{Условная независимость в трехмерном распределение Бернулли}
Определение условной независимости приводится в работе \cite{Lauritzen1996}.
\begin{definition}
    Пусть $(X,Y,Z)^T$ -- дискретный случайный вектор.
    Говорят, что случайные величины $X$ и $Y$ условно независимы при условии $Z$,
    и пишут $X \ci Y \mid Z$, если
    $$
    P(X=x, Y=y \mid Z = z) = P(X=x \mid Z = z) P(Y=y \mid Z = z)
    $$
    при любом $z$ для которого $P(Z=z)>0$.
\end{definition}

Найдем соотношения параметров трехмерного распределения Бернулли, приводящие к условной независимости.
\begin{theorem}\label{thm1}
    Пусть $(X,Y,Z)^T$ -- случайный вектор, имеющий трехмерное распределение Бернулли, в котором $P(Z=0)>0$.
    Случайные величины $X$ и $Y$ условно независимы при условии $Z$ тогда и только тогда, когда
    $p_{00z}p_{11z}=p_{01z}p_{10z}$, где $z=\overline{0,1}$.
\end{theorem}

\begin{proof}
    Пусть $X \ci Y \mid Z$. Значит, для любых $x=\overline{0,1}$, $y=\overline{0,1}$ и $z=\overline{0,1}$ выполнено условие:
    \begin{equation}\label{ci_cond}
        P(X=x, Y=y \mid Z = z) = P(X=x \mid Z = z) P(Y=y \mid Z = z)
    \end{equation}
    После домножения \eqref{ci_cond} на $P(Z=z)^2$ получаем эквивалентное условие:
    \begin{equation}\label{ci_cond2}
        P(X=x,Y=y,Z=z)P(Z=z)=P(X=x,Z=z)P(Y=y,Z=z)
    \end{equation}
    Найдем маргинальное распределение случайной величины $Z$:
    $$
        P(Z=z)=\sum_{x=0}^{1} \sum_{y=0}^{1} p_{xyz} = p_{00z} + p_{01z} + p_{10z} + p_{11z}
    $$
    Найдем маргинальные распределения $(X,Z)^T$ и $(Y,Z)^T$:
    $$
        P(X=x, Z=z) = \sum_{y=0}^{1} p_{xyz} = p_{x0z} + p_{x1z}
    $$
    $$
        P(Y=y, Z=z) = \sum_{x=0}^{1} p_{xyz} = p_{0yz} + p_{1yz}
    $$
    Тогда условие \eqref{ci_cond2} перепишем в следующем виде:
    $$
        p_{xyz} (p_{00z} + p_{01z} + p_{10z} + p_{11z}) = (p_{x0z} + p_{x1z}) (p_{0yz} + p_{1yz})
    $$
    Это условие выполняется для всех $x=\overline{0,1}$, $y=\overline{0,1}$, $z=\overline{0,1}$.
    Пусть $z$ фиксировано.
    Если $x=0$ и $y=0$, то:
    $$
        p_{00z} (p_{00z} + p_{01z} + p_{10z} + p_{11z}) = (p_{00z} + p_{01z}) (p_{00z} + p_{10z})
    $$
    $$
        p_{00z} p_{00z} + p_{00z} p_{01z} + p_{00z} p_{10z} + p_{00z} p_{11z} =
        p_{00z} p_{00z} + p_{00z} p_{10z} + p_{01z} p_{00z} + p_{01z} p_{10z}
    $$
    $$
        p_{00z} p_{11z} = p_{01z} p_{10z}
    $$
    Если $x=0$ и $y=1$, то:
    $$
        p_{01z} (p_{00z} + p_{01z} + p_{10z} + p_{11z}) = (p_{00z} + p_{01z}) (p_{01z} + p_{11z})
    $$
    $$
        p_{01z}p_{00z} + p_{01z}p_{01z} + p_{01z}p_{10z} + p_{01z}p_{11z} =
        p_{00z}p_{01z} + p_{00z} p_{11z} + p_{01z} p_{01z} + p_{01z} p_{11z}
    $$
    $$
        p_{01z}p_{10z}=p_{00z} p_{11z}
    $$
    Если $x=1$ и $y=0$, то:
    $$
        p_{10z} (p_{00z} + p_{01z} + p_{10z} + p_{11z}) = (p_{10z} + p_{11z}) (p_{00z} + p_{10z})
    $$
    $$
        p_{10z} p_{00z} + p_{10z} p_{01z} + p_{10z} p_{10z} + p_{10z} p_{11z} = p_{10z}p_{00z} + p_{10z}p_{10z} + p_{11z}p_{00z} + p_{11z}p_{10z}
    $$
    $$
        p_{10z} p_{01z} = p_{11z}p_{00z}
    $$
    Если $x=1$ и $y=1$, то:
    $$
        p_{11z} (p_{00z} + p_{01z} + p_{10z} + p_{11z}) = (p_{10z} + p_{11z}) (p_{01z} + p_{11z})
    $$
    $$
        p_{11z} p_{00z} + p_{11z} p_{01z} + p_{11z} p_{10z} + p_{11z} p_{11z} = p_{10z} p_{01z} + p_{10z}p_{11z} + p_{11z}p_{01z} +
        p_{11z} p_{11z}
    $$
    $$
        p_{11z} p_{00z} = p_{10z} p_{01z}
    $$
    Таким образом, из условной независимости $X$ и $Y$ при условии $Z$ следует
    $p_{00z}p_{11z}=p_{01z}p_{10z}$, где $z=\overline{0,1}$.

    Доказательство в обратную сторону проводится аналогично.
\end{proof}

\begin{example}
    Пусть $(X,Y,Z)^T$ имеет трехмерное распределение Бернулли с вероятностями
    $p_{000}=0.15$, $p_{001}=0.1$, $p_{010}=0.3$, $p_{011}=0.1$, $p_{100}=0.05$, $p_{101}=0.1$,
    $p_{110}=0.1$, $p_{111}=0.1$.
    Заметим, что:
    $$p_{000}p_{110}=p_{010}p_{100}=0.015$$ $$p_{001}p_{111}=p_{011}p_{101}=0.01$$
    Значит из теоремы \ref{thm1} следует, что $X \ci Y \mid Z$.
\end{example}
