\section*{Заключение}
\addcontentsline{toc}{section}{Заключение}
В настоящей выпускной квалификационной работы 
были получены следующие результаты:
\begin{enumerate}
    \item Сформулирован и доказан критерий условной независимости
    в трехмерном распределении Бернулли.
    \item Доказано, что равенство нулю частного коэффициента корреляции 
    Пирсона $\rho^{XY \cdot Z}$ является
    необходимым, но не достаточным, условием
    условной независимости $X$ и $Y$ при 
    условии $Z$ в трехмерном распределении Бернулли.
    \item Эмпирически, при объеме наблюдений $50 \leq n \leq 300$,
    показано, что тест
    $\varphi^{\text{Partial}}$ является тестом уровня
    $\alpha$ для гипотезы $H^{\text{Partial}}: \rho^{XY\cdot Z}=0$ в трехмерном распределении
    Бернулли.
    \item В экспоненциальной форме записи трехмерного 
    распределения Бернулли найден параметр 
    $\theta = \ln  \left(\dfrac{p_{001}p_{111}p_{010}p_{100}}{p_{011}p_{101}p_{000}p_{110}}\right)$,
    неравенство нулю которого является достаточным 
    условием отсутствия условно независимой пары
    случайных величин в трехмерном распределении Бернулли.
    \item Построен РНМН-тест $\varphi^{\text{Theta}}$
    уровня $\alpha$ проверки гипотезы $H^{\text{Theta}}: \theta~=~0$
    против альтернативы $K^{\text{Theta}}: \theta\neq 0$.
    \item Построен тест $\varphi^{\text{Subsamples}}$ 
    уровня $\alpha$
    проверки гипотезы $H: X \ci Y \mid Z$.
\end{enumerate}
