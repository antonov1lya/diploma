\section*{Заключение}
\addcontentsline{toc}{section}{Заключение}
В настоящей выпускной квалификационной работы 
были получены следующие результаты:
\begin{enumerate}
    \item Сформулирован и доказан критерий условной независимости
    в трехмерном распределении Бернулли.
    \item Доказано, что равенство нулю частного коэффициента корреляции 
    Пирсона $\rho^{XY \cdot Z}$ является
    необходимым, но не достаточным, условием
    условной независимости $X$ и $Y$ при 
    условии $Z$ в трехмерном распределении Бернулли.
    \item Эмпирически, при объеме наблюдений $50 \leq n \leq 300$,
    показано, что тест
    $\varphi^{\text{Partial}}$ является тестом уровня
    $\alpha$ как для гипотезы $H^{\text{Partial}}: \rho^{XY\cdot Z}=0$,
    так и для гипотезы $H: X \ci Y \mid Z$ в трехмерном распределении
    Бернулли.
    \item В экспоненциальной форме записи трехмерного 
    распределения Бернулли найден параметр 
    $\theta = \ln  \left(\dfrac{p_{001}p_{111}p_{010}p_{100}}{p_{011}p_{101}p_{000}p_{110}}\right)$,
    равенство нулю которого является необходимым условием выполнения одного из условий:
    \begin{itemize}
        \item $X \ci Y \mid Z$
        \item $X \ci Z \mid Y$
        \item $Y \ci Z \mid X$
    \end{itemize}
    \item Построен РНМН-тест $\varphi^{\text{Theta}}$
    уровня $\alpha$ проверки гипотезы $H^{\text{Theta}}: \theta~=~0$.
    Показано, что $\varphi^{\text{Theta}}$ является несмещенным тестом 
    уровня $\alpha$ проверки гипотезы $H: X \ci Y \mid Z$,
    однако не является РНМН-тестом проверки этой же гипотезы.
    \item Построен тест $\varphi^{\text{Subsamples}}$ 
    уровня $\alpha$
    проверки гипотезы $H: X \ci Y \mid Z$. Доказано, что 
    этот тест является несмещенным.
\end{enumerate}
Результаты настоящей работы могут быть использованы для задачи идентификации
графической модели с попарным марковским свойством в трехмерном
распределении Бернулли.
