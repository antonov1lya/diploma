\documentclass[a4paper,14pt]{extarticle}
\usepackage{graphicx}

\usepackage[left=30mm, right=10mm, top=20mm, bottom=20mm]{geometry}
% \usepackage[left=25mm, right=15mm, top=15mm, bottom=15mm]{geometry}
\linespread{1.25} 

\usepackage[russian]{babel}

\usepackage{amsmath,amssymb,amsfonts}
\usepackage{amsthm}
\usepackage{hyperref}

\theoremstyle{definition}
\newtheorem{definition}{Определение}[section]
\newtheorem{theorem}{Теорема}[section]
\newtheorem{example}{Пример}[section]
\newtheorem{lemma}{Лемма}[section]

\def\ci{\perp\!\!\!\perp}

\usepackage{hyperref}
\hypersetup{pdfstartview={XYZ null null 1.00}}

\bibliographystyle{gost-numeric.bbx}
\RequirePackage[
    parentracker=true,
    backend=biber,
    bibstyle=gost-numeric,
    citestyle=gost-numeric,
    hyperref=true,
    bibencoding=utf8,
    language=auto,
    autolang=other,
    defernumbers=false,
    doi=false,
    eprint=false,
    isbn=false,
    dashed=false,
    url=false
]{biblatex} 
\addbibresource{refs.bib}
\RequirePackage{bibentry}


\begin{document}

% титульник
\thispagestyle{empty}

\begin{center}
    \textsc{
    ФЕДЕРАЛЬНОЕ ГОСУДАРСТВЕННОЕ АВТОНОМНОЕ\\
    ОБРАЗОВАТЕЛЬНОЕ УЧРЕЖДЕНИЕ\\
    ВЫСШЕГО ОБРАЗОВАНИЯ\\
    «НАЦИОНАЛЬНЫЙ ИССЛЕДОВАТЕЛЬСКИЙ УНИВЕРСИТЕТ\\
    «ВЫСШАЯ ШКОЛА ЭКОНОМИКИ»}
\end{center}

\vfill

\begin{center}
    \textbf{Факультет информатики, математики и компьютерных наук}

    \vspace{20pt}

    \textbf{Программа подготовки бакалавров по направлению \\
    01.03.02 Прикладная математика и информатика}
\end{center}

\vfill

\begin{center}
    \textit{Антонов Илья Витальевич} 
    
    \vspace{20pt}
    
    \textbf{ВЫПУСКНАЯ КВАЛИФИКАЦИОННАЯ РАБОТА}

    \vspace{20pt}
    
    Проверка условной независимости в трехмерном распределении Бернулли
\end{center}

\vfill

\begin{flushright}
Научный руководитель

\vspace{5pt}

д.ф.-м.н., проф.

\vspace{5pt}

П.А. Колданов
\end{flushright}

\vfill

\begin{center}
    Нижний Новгород, 2024
\end{center}
\newpage

% содержание
\tableofcontents
\newpage
\section{Теория проверки условной независимости в трехмерном распределении Бернулли}
\subsection{Условная независимость в трехмерном распределение Бернулли}\label{ci_and_bernoulli}

Определим трехмерное распределение Бернулли \cite{Dai2013, Teugels1990}.
\begin{definition}
    Случайный вектор $(X,Y,Z)^T$ имеет трехмерное распределение Бернулли,
    если множество его возможных значений:
    $$
        \begin{pmatrix}
            0 \\
            0 \\
            0
        \end{pmatrix},
        \begin{pmatrix}
            0 \\
            0 \\
            1
        \end{pmatrix},
        \begin{pmatrix}
            0 \\
            1 \\
            0
        \end{pmatrix}, \ldots, \begin{pmatrix}
            1 \\
            1 \\
            1
        \end{pmatrix}
    $$ и заданы $P(X=x,Y=y,Z=z)=p_{xyz} \geq 0,  \sum_{x=0}^1 \sum_{y=0}^1 \sum_{z=0}^1 p_{xyz} =1$.
\end{definition}
Приведем определение понятия условной независимости \cite{Lauritzen1996}.
\begin{definition}\label{cond_ind_def}
    Пусть $(X,Y,Z)^T$ -- дискретный случайный вектор.
    Говорят, что случайные величины $X$ и $Y$ условно независимы при условии $Z$,
    и пишут $X \ci Y \mid Z$, если:
    $$
    P(X=x, Y=y \mid Z = z) = P(X=x \mid Z = z) P(Y=y \mid Z = z)
    $$
    при любых $x,y$ и $z$ для которого $P(Z=z)>0$.
\end{definition}
Найдем условия на параметры трехмерного распределения 
Бернулли при условной независимости.
\begin{theorem}\label{thm1}
    Пусть $(X,Y,Z)^T$ -- случайный вектор, имеющий трехмерное распределение Бернулли, в котором $P(Z=0)>0$.
    Случайные величины $X$ и $Y$ условно независимы при условии $Z$ тогда и только тогда, когда
    $$p_{00z}p_{11z}=p_{01z}p_{10z}$$ для всех $z=\overline{0,1}$.
\end{theorem}
\begin{proof}
    Пусть $X \ci Y \mid Z$. Значит, для любых $x=\overline{0,1}$, $y=\overline{0,1}$ и $z=\overline{0,1}$ выполнено условие:
    \begin{equation}\label{ci_cond}
        P(X=x, Y=y \mid Z = z) = P(X=x \mid Z = z) P(Y=y \mid Z = z)
    \end{equation}
    После домножения \eqref{ci_cond} на $P(Z=z)^2>0$ получаем эквивалентное условие:
    \begin{equation}\label{ci_cond2}
        P(X=x,Y=y,Z=z)P(Z=z)=P(X=x,Z=z)P(Y=y,Z=z)
    \end{equation}
    Найдем следующие вероятности:
    $$
        P(X=x, Z=z) = p_{x0z} + p_{x1z},\;  P(Y=y, Z=z) = p_{0yz} + p_{1yz}
    $$
    $$
        P(Z=z)= p_{00z} + p_{01z} + p_{10z} + p_{11z}
    $$
    Тогда условие \eqref{ci_cond2} перепишем в следующем виде:
    $$
        p_{xyz} (p_{00z} + p_{01z} + p_{10z} + p_{11z}) = (p_{x0z} + p_{x1z}) (p_{0yz} + p_{1yz})
    $$
    Это условие выполняется для всех $x=\overline{0,1}$, $y=\overline{0,1}$, $z=\overline{0,1}$.
    Пусть $z$ фиксировано.
    Если $x=0$ и $y=0$, то:
    $$
        p_{00z} (p_{00z} + p_{01z} + p_{10z} + p_{11z}) = (p_{00z} + p_{01z}) (p_{00z} + p_{10z})
    \Leftrightarrow
        p_{00z} p_{11z} = p_{01z} p_{10z}
    $$
    Если $x=0$ и $y=1$, то:
    $$
        p_{01z} (p_{00z} + p_{01z} + p_{10z} + p_{11z}) = (p_{00z} + p_{01z}) (p_{01z} + p_{11z})
    \Leftrightarrow
    p_{00z} p_{11z} = p_{01z} p_{10z}
    $$
    Если $x=1$ и $y=0$, то:
    $$
        p_{10z} (p_{00z} + p_{01z} + p_{10z} + p_{11z}) = (p_{10z} + p_{11z}) (p_{00z} + p_{10z})
    \Leftrightarrow
    p_{00z} p_{11z} = p_{01z} p_{10z}
    $$
    Если $x=1$ и $y=1$, то:
    $$
        p_{11z} (p_{00z} + p_{01z} + p_{10z} + p_{11z}) = (p_{10z} + p_{11z}) (p_{01z} + p_{11z})
    \Leftrightarrow
    p_{00z} p_{11z} = p_{01z} p_{10z}
    $$
    Таким образом, из $X \ci Y \mid Z$ следует
    $p_{00z}p_{11z}=p_{01z}p_{10z}$ для всех $z=\overline{0,1}$.

    Поскольку в вышеприведенных рассуждениях все переходы равносильные,
    мы также доказали, что из из условия $p_{00z}p_{11z}=p_{01z}p_{10z}$ для всех $z=\overline{0,1}$
    следует $X \ci Y \mid Z$.
\end{proof}
Покажем, что существует случайный вектор $(X,Y,Z)^T$ с трехмерным распределением Бернулли, в котором $X \ci Y \mid Z$.
\begin{example}
    Пусть $(X,Y,Z)^T$ имеет трехмерное распределение Бернулли с вероятностями
    $p_{000}=0.15$, $p_{001}=0.1$, $p_{010}=0.3$, $p_{011}=0.1$, $p_{100}=0.05$, $p_{101}=0.1$,
    $p_{110}=0.1$, $p_{111}=0.1$.
    Заметим, что:
    $$p_{000}p_{110}=p_{010}p_{100}=0.015$$ $$p_{001}p_{111}=p_{011}p_{101}=0.01$$
    Следовательно из \autoref{thm1} следует, что $X \ci Y \mid Z$.
\end{example}

\begin{centering}
    \subsection{Частный коэффициент корреляции Пирсона в трехмерном распределении Бернулли}\label{partial_section}
\end{centering}
% Согласно \cite{Anderson2003} в трехмерном нормальном распределении случайные величины 
% $X$ и $Y$ условно независимы при условии $Z$ тогда и только тогда, когда частный коэффициент корреляции Пирсона
% между $X$ и $Y$ принимает нулевое значение.
% Проверим, сохраняется ли это свойство в трехмерном распределении Бернулли.
В данном разделе исследуем свойства частного коэффициента корреляции Пирсона
в трехмерном распределении Бернулли.

Для случайного вектора $(X,Y,Z)^T$ определим ковариационную матрицу:
$$\Sigma =
    \begin{pmatrix}
        \sigma_{XX} & \sigma_{XY} & \sigma_{XZ} \\
        \sigma_{YX} & \sigma_{YY} & \sigma_{YZ} \\
        \sigma_{ZX} & \sigma_{ZY} & \sigma_{ZZ}
    \end{pmatrix}
$$
где $\sigma_{XY}=E[(X-EX)(Y-EY)]$.
Остатками от $X$ и $Y$ при регрессии на $Z$ называются случайные величины:
$$
X^{\prime}=(X-EX)-\dfrac{\sigma_{XZ}}{\sigma_{ZZ}}(Z-EZ)
$$
$$
Y^{\prime}=(Y-EY)-\dfrac{\sigma_{YZ}}{\sigma_{ZZ}}(Z-EZ)
$$
Согласно работе \cite{Cramér1946}, частный коэффициент корреляции Пирсона определяется как коэффициент
корреляции Пирсона между остатками, другими словами:
$$
\rho^{XY \cdot Z}=\dfrac{E(X^{\prime} Y^{\prime})}{\sqrt{E(X^{\prime})^2 E(Y^{\prime})^2}}
$$
Приведем соотношения, которые справедливы
для $\rho^{XY \cdot Z}$ в произвольном распределении.
\begin{lemma}
    Пусть $(X,Y,Z)^T$ -- произвольный случайный вектор. Тогда:
    $$
    \rho^{XY \cdot Z}=\dfrac{\sigma_{XY} \sigma_{ZZ} - \sigma_{XZ} \sigma_{YZ}}{\sqrt{\sigma_{XX}\sigma_{ZZ}-
        \sigma_{XZ}^2}\sqrt{\sigma_{YY}\sigma_{ZZ}-\sigma_{YZ}^2}}=\dfrac{\rho_{XY}-\rho_{XZ}\rho_{YZ}}{\sqrt{1-\rho_{XZ}^2}\sqrt{1-\rho_{YZ}^2}}
    $$
    где $\rho_{XY}, \; \rho_{XZ}, \; \rho_{YZ}$ -- коэффициент корреляции Пирсона
        между случайными величинами $X$ и $Y$, $X$ и $Z$, $Y$ и $Z$
        соответственно.
\end{lemma}
\begin{proof}
    $$E(X^\prime Y^\prime)=E[((X-EX)-\dfrac{\sigma_{XZ}}{\sigma_{ZZ}}(Z-EZ))((Y-EY)-\dfrac{\sigma_{YZ}}{\sigma_{ZZ}}(Z-EZ))]=$$
    $$=\sigma_{XY}-\dfrac{\sigma_{YZ}}{\sigma_{ZZ}}\sigma_{XZ}-\dfrac{\sigma_{XZ}}{\sigma_{ZZ}}\sigma_{YZ}+
    \dfrac{\sigma_{XZ}\sigma_{YZ}}{\sigma_{ZZ}\sigma_{ZZ}}\sigma_{ZZ}=\dfrac{\sigma_{XY}\sigma_{ZZ}-\sigma_{XZ}\sigma_{YZ}}{\sigma_{ZZ}}$$
    Тогда:
    $$
    \rho^{XY \cdot Z}=\dfrac{E(X^{\prime} Y^{\prime})}{\sqrt{E(X^{\prime})^2 E(Y^{\prime})^2}}=\dfrac{\dfrac{\sigma_{XY}\sigma_{ZZ}-\sigma_{XZ}\sigma_{YZ}}{\sigma_{ZZ}}}
    {\sqrt{
        \dfrac{\sigma_{XX}\sigma_{ZZ}-\sigma_{XZ}\sigma_{XZ}}{\sigma_{ZZ}}
    }\sqrt{
        \dfrac{\sigma_{YY}\sigma_{ZZ}-\sigma_{YZ}\sigma_{YZ}}{\sigma_{ZZ}}
    }}=
    $$
    $$
    =\dfrac{\sigma_{XY} \sigma_{ZZ} - \sigma_{XZ} \sigma_{YZ}}{\sqrt{\sigma_{XX}\sigma_{ZZ}-
    \sigma_{XZ}^2}\sqrt{\sigma_{YY}\sigma_{ZZ}-\sigma_{YZ}^2}}=
    $$
    $$
    =\dfrac{\sigma_{ZZ}\sqrt{\sigma_{XX}\sigma_{YY}}
    \left(\dfrac{\sigma_{XY}}{\sqrt{\sigma_{XX}\sigma_{YY}}} 
    - \dfrac{\sigma_{XZ}}{\sqrt{\sigma_{XX}\sigma_{ZZ}}} \dfrac{\sigma_{YZ}}{\sqrt{\sigma_{YY}\sigma_{ZZ}}}\right)}
    {\sqrt{\sigma_{XX}\sigma_{ZZ}-
    \sigma_{XZ}^2}\sqrt{\sigma_{YY}\sigma_{ZZ}-\sigma_{YZ}^2}}=
    $$
    $$
    =\dfrac{\rho_{XY}-\rho_{XZ}\rho_{YZ}}{\sqrt{1-\dfrac{\sigma_{XZ}^2}{\sigma_{XX}\sigma_{ZZ}}}
    \sqrt{1-\dfrac{\sigma_{YZ}^2}{\sigma_{YY}\sigma_{ZZ}}}}
    =\dfrac{\rho_{XY}-\rho_{XZ}\rho_{YZ}}{\sqrt{1-\rho_{XZ}^2}\sqrt{1-\rho_{YZ}^2}}
    $$
\end{proof}

Для дальнейших рассуждений примем следующие обозначения: 
$$p_{x**}=P(X=x), \; p_{*y*}=P(Y=y), \; p_{**z}=P(Z=z)$$ 
$$  p_{xy*}=P(X=x, Y=y),\; p_{x*z}=P(X=x, Z=z), \; p_{*yz}=P(Y=y, Z=z)$$
Найдем значение выражения $\sigma_{XY} \sigma_{ZZ} - \sigma_{XZ} \sigma_{YZ}$ в трехмерном распределении Бернулли.
\begin{lemma}\label{partial_cov}
    Пусть $(X,Y,Z)^T$ -- случайный вектор, имеющий трехмерное распределение Бернулли. Тогда:
    $$\sigma_{XY} \sigma_{ZZ} - \sigma_{XZ} \sigma_{YZ} = p_{**0}(p_{001}p_{111}-p_{011}p_{101}) + p_{**1} (p_{000}p_{110}-p_{010}p_{100})$$
\end{lemma}
\begin{proof}
    Легко проверить, что $\sigma_{ZZ}= p_{**1}(1-p_{**1})$. Найдем соотношение для $\sigma_{XY}$. Воспользуемся формулой $\sigma_{XY}=E(X Y)-E(X)E(Y)$.
    $$E(X Y) = 1 \cdot p_{11*} + 0 \cdot (p_{00*} + p_{01*} + p_{10*})=p_{11*}$$
    Таким образом, $\sigma_{XY}=p_{11*}-p_{1**}p_{*1*}$. Аналогично, $\sigma_{XZ}=p_{1*1}-p_{1**}p_{**1}$ и $\sigma_{YZ}=p_{*11}-p_{*1*}p_{**1}$.
    Преобразуем выражение $\sigma_{XY} \sigma_{ZZ} - \sigma_{XZ} \sigma_{YZ} =$
    $$
        = (p_{11*}-p_{1**}p_{*1*}) p_{**1}(1-p_{**1})
        -(p_{1*1}-p_{1**}p_{**1})(p_{*11}-p_{*1*}p_{**1})=
    $$
    $$
        = p_{11*}p_{**1} - p_{11*}p_{**1}p_{**1} - p_{1**}p_{*1*}p_{**1} 
        -p_{1*1}p_{*11}+p_{1*1}p_{*1*}p_{**1}+p_{1**}p_{**1}p_{*11} =
    $$
    $$
        =(p_{111}p_{**1}+p_{110}p_{**1}) - p_{11*}p_{**1}p_{**1} - p_{1**}p_{*1*}p_{**1} -
    $$ $$
        -p_{1*1}p_{*11}+p_{1*1}p_{*1*}p_{**1}+p_{1**}p_{**1}p_{*11}=
    $$
    \begin{equation}\label{some_step}
        =(p_{111}p_{**1}-p_{1*1}p_{*11})+p_{**1}(p_{110}-p_{11*}p_{**1} - p_{1**}p_{*1*} + p_{1*1}p_{*1*} + p_{1**}p_{*11})
    \end{equation}
    Заметим, что:\\
    1) $
        p_{110}-p_{11*}p_{**1}=p_{110}-p_{110}p_{**1}-p_{111}p_{**1}=
        p_{110}(1-p_{**1})-p_{111}p_{**1}=
    $
    $ 
        =p_{110}p_{**0}-p_{111}p_{**1}
    $\\
    2) $
        -p_{1**}p_{*1*} + p_{1*1}p_{*1*} + p_{1**}p_{*11}=
    $
    $
        -(p_{1*0}+p_{1*1})(p_{*10}+p_{*11})+p_{1*1}(p_{*10}+p_{*11}) + $
        $+(p_{1*0}+p_{1*1})p_{*11}=
        -p_{1*0}p_{*10}+p_{1*1}p_{*11}
    $\\
    Учитывая вышеприведенные соотношения, запишем \eqref{some_step}:
    $$
    (p_{111}p_{**1}-p_{1*1}p_{*11})+p_{**1}((p_{110}p_{**0}-p_{1*0}p_{*10})-(p_{111}p_{**1}-p_{1*1}p_{*11}))=
    $$
    $$
        =(1-p_{**1})(p_{111}p_{**1}-p_{1*1}p_{*11})+p_{**1}(p_{110}p_{**0}-p_{1*0}p_{*10})=
    $$
    \begin{equation}\label{some_step_2}
        =p_{**0}(p_{111}p_{**1}-p_{1*1}p_{*11})+p_{**1}(p_{110}p_{**0}-p_{1*0}p_{*10})
    \end{equation}
    Также заметим, что:\\
    1) $
        p_{111}p_{**1}-p_{1*1}p_{*11} = p_{111}(p_{001}+p_{011}+p_{101}+p_{111})-
        (p_{101}+p_{111})(p_{011}+p_{111})=
    $
    $
        = p_{001}p_{111}-p_{011}p_{101}
    $
    \\
    2) $
        p_{110}p_{**0}-p_{1*0}p_{*10}=
        p_{110}(p_{000}+p_{010}+p_{100}+p_{110})-(p_{100}+p_{110})(p_{010}+p_{110})=
    $
    $
        =p_{000}p_{110}-p_{010}p_{100}
    $\\
    Подставляя преобразованные выражения в \eqref{some_step_2} имеем:
    $$
    \sigma_{XY} \sigma_{ZZ} - \sigma_{XZ} \sigma_{YZ} = p_{**0}(p_{001}p_{111}-p_{011}p_{101}) + p_{**1} (p_{000}p_{110}-p_{010}p_{100})
    $$
\end{proof}
Вышеприведенное соотношение для $\sigma_{XY} \sigma_{ZZ} - \sigma_{XZ} \sigma_{YZ}$ позволяет доказать следующую теорему.
\begin{theorem}\label{1.2}
    Пусть $(X,Y,Z)^T$ -- случайный вектор, имеющий трехмерное распределение Бернулли.
    Если $X \ci Y \mid Z$, то $\rho^{XY \cdot Z}=0$.
\end{theorem}
\begin{proof}
    Пусть $X \ci Y \mid Z$. Тогда по \autoref{thm1}:
    $p_{000}p_{110}=p_{010}p_{100}$ и 
    $p_{001}p_{111}=p_{011}p_{101}$.
    Используя эти соотношения в числителе частного коэффициента корреляции Пирсона и учитывая \autoref{partial_cov}, имеем:
    $$
    \sigma_{XY} \sigma_{ZZ} - \sigma_{XZ} \sigma_{YZ}=p_{**0}(p_{001}p_{111}-p_{011}p_{101}) + p_{**1} (p_{000}p_{110}-p_{010}p_{100})= 0
    $$
    Следовательно, $\rho^{XY \cdot Z}=0$.
\end{proof}
Таким образом, равенство нулю частного коэффициента корреляции Пирсона является необходимым условием условной независимости.
Однако, это условие не является достаточным, так как в обратную сторону \autoref{1.2} неверна. Приведем контрпример.
\begin{example}
    Пусть $p_{000}=0.15$, $p_{001}=0.1$, $p_{010}=0.1$, $p_{011}=0.15$, $p_{100}=0.1$, $p_{101}=0.15$, $p_{110}=0.15$, $p_{111}=0.1$.
    Тогда $p_{**0}=0.5$, $p_{**1}=0.5$ и
    $$\sigma_{XY} \sigma_{ZZ} - \sigma_{XZ} \sigma_{YZ} = p_{**1}(p_{000}p_{110}-p_{010}p_{100}) + p_{**0}(p_{001}p_{111}-p_{011}p_{101})=$$
    $$=0.5 \cdot (0.15 \cdot 0.15 - 0.1 \cdot 0.1) + 0.5 \cdot (0.1 \cdot 0.1 - 0.15 \cdot 0.15) = 0$$
    А значит $\rho^{XY\cdot X}=0$.
    Однако, случайные величины $X$ и $Y$ условно зависимы при условии $Z$ поскольку:
    $$
        p_{000}p_{110}-p_{010}p_{100}=0.15 \cdot 0.15 - 0.1 \cdot 0.1 = 0.0125 \neq 0
    $$
    $$
        p_{001}p_{111}-p_{011}p_{101}=0.1 \cdot 0.1 - 0.15 \cdot 0.15 = -0.0125 \neq 0
    $$
\end{example}

Для оценки частного коэффициента корреляции Пирсона 
можно использовать
выборочный частный коэффициент корреляции Пирсона:
$$r^{XY\cdot Z}=\dfrac{r_{XY}-r_{XZ}r_{YZ}}{\sqrt{1-r_{XY}^2}\sqrt{1-r_{YZ}^2}}$$
где $r_{XY}$, $r_{XZ}$, $r_{YZ}$ -- выборочный коэффициент корреляции Пирсона
между случайными величинами $X$ и $Y$, $X$ и $Z$, $Y$ и $Z$
соответственно.  
Известно \cite{Anderson2003}, что 
в трехмерном нормальном распределении
при истинности гипотезы $H^{\text{Partial}}: \rho^{XY \cdot Z}=0$ статистика
$$
T^{\text{Partial}}=\sqrt{n-3} \dfrac{r^{XY \cdot Z}}{\sqrt{1-(r^{XY \cdot Z})^2}}
$$
имеет распределение Стьюдента с $n-3$ степенями свободы, 
где $n$ -- количество наблюдений.
Тогда тест уровня $\alpha$ проверки гипотезы $H^{\text{Partial}}:\rho^{XY\cdot Z}=0$
против альтернативы $K^{\text{Partial}}:\rho^{XY\cdot Z}\neq 0$
определяется как:
$$
\varphi^{\text{Partial}}(t) = \begin{cases}
    1, \; |t|>C \\ 
    0, \; |t|\leq C
\end{cases}
$$
где константа $C$ удовлетворяет уравнению
$P(T^{\text{Partial}}>C)=1-\alpha/2$.
% Предположим, что $\varphi^{\text{Partial}}$ является тестом
% уровня $\alpha$ для проверки гипотезы $\rho^{XY \cdot Z}=0$ в трехмерном распределении Бернулли. 
% Поскольку из $X \ci Y \mid Z$ следует $\rho^{XY\cdot Z}=0$, то 
% тест $\varphi^{\text{Partial}}$ также является тестом
% уровня $\alpha$ для проверки гипотезы $h: X \ci Y \mid Z$.
В разделе \ref{numerical_exp} с помощью численных экспериментов 
будет проверено контролирует ли тест $\varphi^{\text{Partial}}$
уровень значимости для гипотезы $H^{\text{Partial}}:\rho^{XY\cdot Z}=0$ в трехмерном распределении Бернулли.
% В настоящей работе с помощью численных экспериментов
% будет проверено контролирует ли тест 
% $\varphi^{\text{Partial}}$ вероятность ошибки первого рода
% в трехмерном распределении Бернулли.
% \begin{lemma}
%     Пусть $\Sigma$ -- ковариационная матрица с элементами
%     $$\Sigma =
%     \begin{pmatrix}
%         \sigma_{XX} & \sigma_{XY} & \sigma_{XZ} \\
%         \sigma_{YX} & \sigma_{YY} & \sigma_{YZ} \\
%         \sigma_{ZX} & \sigma_{ZY} & \sigma_{ZZ}
%     \end{pmatrix}
%     $$
%     а $\Sigma^{-1}$ обратная ковариационная матрица с элементами
%     $$\Sigma^{-1} =
%     \begin{pmatrix}
%         \sigma^{XX} & \sigma^{XY} & \sigma^{XZ} \\
%         \sigma^{YX} & \sigma^{YY} & \sigma^{YZ} \\
%         \sigma^{ZX} & \sigma^{ZY} & \sigma^{ZZ}
%     \end{pmatrix}
%     $$
%     Тогда для частного коэффициента корреляции справедливо:
%     $$
%     \rho^{XY \cdot Z}=-\dfrac{\sigma^{XY}}{\sqrt{\sigma^{XX}\sigma^{YY}}}
%     $$
% \end{lemma}
% \begin{proof}
%     Воспользуемся следующими соотношениями для элементов обратной матрицы:
%     $$
%     \sigma^{XY}=\dfrac{-1}{det(\Sigma)}\begin{vmatrix}
%     \sigma_{XY} & \sigma_{XZ}\\
%     \sigma_{YZ} & \sigma_{ZZ}
%     \end{vmatrix}=\dfrac{-(\sigma_{XY}\sigma_{ZZ}-\sigma_{XZ}\sigma_{YZ})}{det(\Sigma)}
%     $$

%     $$
%     \sigma^{XX}=\dfrac{1}{det(\Sigma)}\begin{vmatrix}
%     \sigma_{YY} & \sigma_{YZ}\\
%     \sigma_{YZ} & \sigma_{ZZ}
%     \end{vmatrix}
%     = \dfrac{\sigma_{YY}\sigma_{ZZ}-\sigma_{YZ}^2}{det(\Sigma)}
%     $$

%     $$
%     \sigma^{YY}=\dfrac{1}{det(\Sigma)}\begin{vmatrix}
%     \sigma_{XX} & \sigma_{XZ}\\
%     \sigma_{XZ} & \sigma_{ZZ}
%     \end{vmatrix}
%     = \dfrac{\sigma_{XX}\sigma_{ZZ}-\sigma_{XZ}^2}{det(\Sigma)}
%     $$
%     Тогда:
%     $$
%     -\dfrac{\sigma^{XY}}{\sqrt{\sigma^{XX}\sigma^{YY}}}=-\dfrac{
%         \dfrac{-(\sigma_{XY}\sigma_{ZZ}-\sigma_{XZ}\sigma_{YZ})}{det(\Sigma)}
%     }{
%         \sqrt{\dfrac{\sigma_{XX}\sigma_{ZZ}-\sigma_{XZ}^2}{det(\Sigma)}}
%         \sqrt{\dfrac{\sigma_{YY}\sigma_{ZZ}-\sigma_{YZ}^2}{det(\Sigma)}}
%     }=
%     $$
%     $$
%     = \dfrac{\sigma_{XY}\sigma_{ZZ}-\sigma_{XZ}\sigma_{YZ}}{
%         \sqrt{\sigma_{XX}\sigma_{ZZ}-\sigma_{XZ}^2}
%         \sqrt{\sigma_{YY}\sigma_{ZZ}-\sigma_{YZ}^2}
%     }=\rho^{XY \cdot Z}
%     $$
% \end{proof}
\begin{centering}
    \subsection{Тест на параметр трехмерного распределения Бернулли в экспоненциальной форме}\label{expon_form_section}
\end{centering}
Покажем вид трехмерного распределения Бернулли в экпоненциальной форме:
$$
    P(X=x,Y=y,Z=z) = p_{000}^{(1-x)(1-y)(1-z)} \ldots p_{111}^{x y z} =
$$
$$
    =
        \exp \Biggl\{\ln(p_{000}) +   \ln  \left(\dfrac{p_{001}p_{111}p_{010}p_{100}}{p_{011}p_{101}p_{000}p_{110}}\right)  xyz
     +   \ln\left(\dfrac{p_{100}}{p_{000}}\right) x +   \ln\left(\dfrac{p_{010}}{p_{000}}\right) y + $$
    $$    
    +  \ln\left(\dfrac{p_{001}}{p_{000}}\right) z
        +  \ln \left(\dfrac{p_{000}p_{110}}{p_{010}p_{100}}\right) xy +
         \ln \left(\dfrac{p_{000}p_{101}}{p_{001}p_{100}}\right) xz +
         \ln \left(\dfrac{p_{000}p_{011}}{p_{001}p_{010}}\right) yz \Biggr\}
    $$
Среди параметров, стоящих при статистиках $xyz,x,y,z,xy,xz,yz$ выделим параметр, связанный с условной независимостью.
\begin{theorem}
    Пусть $(X,Y,Z)^T$ -- случайный вектор, имеющий трехмерное распределение Бернулли, и 
    $\theta = \ln  \left(\dfrac{p_{001}p_{111}p_{010}p_{100}}{p_{011}p_{101}p_{000}p_{110}}\right)$.
    Если выполнено одно из условий:
    \begin{itemize}
        \item $X \ci Y \mid Z$
        \item $X \ci Z \mid Y$
        \item $Y \ci Z \mid X$
    \end{itemize}
    то параметр $\theta$ принимает значение $0$.
    \end{theorem}
    
    \begin{proof}
        Результаты теоремы $\ref{thm1}$ можно обобщить следующим образом:
        $$
        X \ci Z \mid Y \Leftrightarrow p_{000}p_{101}=p_{001}p_{100} \text{ и } p_{010}p_{111}=p_{011}p_{110}
        $$
        $$
        Y \ci Z \mid X \Leftrightarrow p_{000}p_{011}=p_{001}p_{010} \text { и } p_{100}p_{111}=p_{101}p_{110}
        $$
        \begin{enumerate}
            \item Пусть $X \ci Y \mid Z$, тогда по теореме \ref{thm1} выполнено:
            $p_{000}p_{110}=p_{010}p_{100}$ и  $p_{001}p_{111}=p_{011}p_{101}$. Отсюда следует, что
            $\theta=\ln(1)=0$.
            \item Пусть $X \ci Z \mid Y$, тогда из вышеприведенных соображений
            $p_{000}p_{101}=p_{001}p_{100}$ и $p_{010}p_{111}=p_{011}p_{110}$. Отсюда следует, что
            $\theta=\ln(1)=0$.
            \item Пусть $Y \ci Z \mid X$, тогда из вышеприведенных соображений
            $p_{000}p_{011}=p_{001}p_{010}$ и $p_{100}p_{111}=p_{101}p_{110}$. Отсюда следует, что
            $\theta=\ln(1)=0$.
        \end{enumerate}
    \end{proof}

    Таким образом, нулевое значение параметра
    $\theta$ является необходимым условием наличия условно независимой пары
    случайных величин в трехмерном распределении Бернулли.
    Для проверки гипотезы о равенстве параметра $\theta$ нулю используем 
    теорию РНМН тестов \cite{Lehmann1986} в многопараметрическом 
    экспоненциальном семействе. Пусть
    $$
        \begin{pmatrix}
            X_1 \\
            Y_1 \\
            Z_1
        \end{pmatrix},
        \begin{pmatrix}
            X_2 \\
            Y_2 \\
            Z_2
        \end{pmatrix}, \ldots,
        \begin{pmatrix}
            X_n \\
            Y_n \\
            Z_n
        \end{pmatrix}
    $$ повторная выборка из распределения случайного вектора $(X,Y,Z)^T$.
    Совместное распределение повторной выборки имеет вид:
    $$
    P(X_1=x_1,Y_1=y_1,Z_1=z_1,\ldots,X_n=x_n,Y_n=y_n,Z_n=z_n)=
    $$
    $$
    =\prod_{i=1}^n P(X_i=x_i,Y_i=y_i,Z_i=z_i) =
    $$
    $$
     =\exp \Biggl\{\ln(p_{000})n + \ln  \left(\dfrac{p_{001}p_{111}p_{010}p_{100}}{p_{011}p_{101}p_{000}p_{110}}\right) \sum_{i=1}^n x_i y_i z_i +$$
        $$ +
            \ln\left(\dfrac{p_{100}}{p_{000}}\right) \sum_{i=1}^{n} x_i + \ln\left(\dfrac{p_{010}}{p_{000}}\right) \sum_{i=1}^{n} y_i +
            \ln\left(\dfrac{p_{001}}{p_{000}}\right) \sum_{i=1}^{n} z_i +
        $$
        $$
            +\ln \left(\dfrac{p_{000}p_{110}}{p_{010}p_{100}}\right) \sum_{i=1}^n x_i y_i +
            \ln \left(\dfrac{p_{000}p_{101}}{p_{001}p_{100}}\right) \sum_{i=1}^n x_i z_i +
            \ln \left(\dfrac{p_{000}p_{011}}{p_{001}p_{010}}\right) \sum_{i=1}^n y_i z_i \Biggr\}
        $$
    Пусть 
    $$
        U = \sum_{i=1}^n X_i Y_i Z_i, \;
        T_1 = \sum_{i=1}^n X_i Y_i, \;
        T_2 = \sum_{i=1}^n X_i Z_i, \;
    $$
    $$
        T_3 = \sum_{i=1}^n Y_i Z_i, \;
        T_4 = \sum_{i=1}^n X_i, \;
        T_5 = \sum_{i=1}^n Y_i, \;
        T_6 = \sum_{i=1}^n Z_i \;
    $$
    Обозначим $T=(T_1,\ldots,T_6)$, $t=(t_1,\ldots,t_6)$, $\theta_0=0$.
    Тогда согласно \cite{Lehmann1986} РНМН тест уровня $\alpha$ проверки гипотезы $\theta=\theta_0$ против альтернативы $\theta \neq \theta_0$ имеет вид:
    $$
    \varphi^{\text{Theta}}(u,t)=\begin{cases}
        1, \; u<C_1(t) \text{ или } u>C_2(t)\\
        \gamma_i, \; u=C_i(t), \; i=1,2\\
        0, \; C_1(t)<u<C_2(t)
    \end{cases}
    $$
    где константы $C_i$ и $\gamma_i$ определяются из системы уравнений:
    $$
    \begin{cases}
        E_{\theta_0}[\varphi^{\text{Theta}}(U,T) \mid T=t]=\alpha \\
        E_{\theta_0}[U\varphi^{\text{Theta}}(U,T) \mid T=t]=\alpha E_{\theta_0}[U \mid T=t]
    \end{cases}
    $$
    Приведем распределение статистики $U$ при условии $T=t$.      
    \begin{lemma}\label{u_dist}
        Пусть $k_1(u)=u$, $k_2(u)=t_1-u$, $k_3(u)=t_2-u$, $k_4(u)=t_3-u$, $k_5(u)=t_4-t_1-t_2+u$, $k_6(u)=t_5-t_1-t_3+u$,
        $k_7(u)=t_6 - t_2 - t_3 + u$, $k_8(u)=n-u+t_1+t_2+t_3-t_4-t_5-t_6$.
        Тогда
        $$P_{\theta_0}(U=u \mid T=t)=\dfrac{(\prod_{i=1}^8 k_i(u)!)^{-1}}
            {\sum_{s\in \mathcal{D}} (\prod_{i=1}^8 k_i(s)!)^{-1}}$$
        где $\mathcal{D}=\{s \in \mathbb{Z}: 0\leq k_i(s) \leq n \text{ для всех } i=1\ldots,8\}$.
    \end{lemma}
    \begin{proof}
        Найдем совместное распределение статистик $(U,T_1,\ldots,T_6)$:
        $$
            P(U=u,T=t)=P(U=u, T_1=t_1, T_2=t_2, T_3=t_3, T_4=t_4, T_5=t_5, T_6=t_6)=
        $$
        $$
            =P\biggl(\sum_{i=1}^n X_i Y_i Z_i=u, \sum_{i=1}^n X_i Y_i=t_1, \sum_{i=1}^n X_i Z_i=t_2,\sum_{i=1}^n Y_i Z_i=t_3,
            $$
            $$
            \sum_{i=1}^n X_i=t_4,\sum_{i=1}^n Y_i=t_5, \sum_{i=1}^n Z_i=t_6\biggr)=
        $$
        $$
            =P\biggl(\sum_{i=1}^n X_i Y_i Z_i=u, \sum_{i=1}^n X_i Y_i (1- Z_i)=t_1-u, \sum_{i=1}^n X_i (1-Y_i) Z_i=t_2-u,
        $$
        $$
            \sum_{i=1}^n (1-X_i) Y_i Z_i=t_3-u,
            \sum_{i=1}^{n} X_i(1-Y_i)(1-Z_i)=t_4-t_1-t_2+u,
        $$
        $$
            \sum_{i=1}^{n} (1-X_i)Y_i(1-Z_i)=t_5-t_1-t_3+u,
            \sum_{i=1}^{n} (1-X_i)(1-Y_i)Z_i = t_6 - t_2 - t_3 + u,
        $$
        $$
            \sum_{i=1}^n (1-X_i)(1-Y_i)(1-Z_i)=n-u+t_1+t_2+t_3-t_4-t_5-t_6\biggr)
            = \frac{n!}{\prod_{i=1}^8 k_i(u)!} \times
        $$
        $$    
        \times p_{111}^u p_{110}^{t_1-u} p_{101}^{t_2-u} p_{011}^{t_3-u}
            p_{100}^{t_4-t_1-t_2+u} p_{010}^{t_5-t_1-t_3+u} p_{001}^{t_6 - t_2 - t_3 + u} 
            p_{000}^{n-u+t_1+t_2+t_3-t_4-t_5-t_6}
            $$
        Тогда условное распределение статистики $U$ при условии $T=t$ можно записать как:
        $$P(U=u \mid T=t)=\dfrac{P(U=u,T=t)}{P(T=t)}=
        \dfrac{P(U=u,T=t)}{\sum_{s \in \mathcal{D}} P(U=s,T=t)}=
        $$
        $$
        =\dfrac{(\prod_{i=1}^8 k_i(u)!)^{-1} \left(\dfrac{p_{001}p_{010}p_{100}p_{111}}{p_{000}p_{011}p_{101}p_{110}}\right)^u}
            {\sum_{s\in \mathcal{D}} (\prod_{i=1}^8 k_i(s)!)^{-1} \left(\dfrac{p_{001}p_{010}p_{100}p_{111}}{p_{000}p_{011}p_{101}p_{110}}\right)^s}$$
        При истинности гипотезы $\theta=\theta_0$ параметр $\dfrac{p_{001}p_{010}p_{100}p_{111}}{p_{000}p_{011}p_{101}p_{110}}=1$. Следовательно:
        $$P_{\theta_0}(U=u \mid T=t)=\dfrac{(\prod_{i=1}^8 k_i(u)!)^{-1}}
            {\sum_{s\in \mathcal{D}} (\prod_{i=1}^8 k_i(s)!)^{-1}}$$
    \end{proof}
    Так как $\varphi^{\text{Theta}}$ является тестом уровня
    $\alpha$ для проверки гипотезы $\theta=\theta_0$, где $\theta_0=0$,
    и из $X \ci Y \mid Z$ следует $\theta=0$, то 
    $\varphi^{\text{Theta}}$ является тестом уровня $\alpha$ для проверки
    гипотезы $h: X \ci Y \mid Z$.
    % Условную вероятность из леммы \ref{u_dist} можно эффективно вычислить на ЭВМ.
    % Пусть $f(i)=\sum_{j=1}^{i} \ln(j)$. Тогда значение условной вероятности представится в виде:
    % $$
    % \dfrac{\frac{1}{\prod_{i=1}^8 k_i(u)!}}{\sum_{s \in D} \frac{1}{\prod_{i=1}^8 k_i(s)!}}=
    % \dfrac{\exp \biggl\{ \ln \biggl( \frac{1}{\prod_{i=1}^8 k_i(u)!} \biggr) \biggr\}}
    % {\sum_{s \in D} \exp \biggl\{ \ln \biggl( \frac{1}{\prod_{i=1}^8 k_i(s)!} \biggr) \biggr\}}
    % = \dfrac{\exp \biggl\{ -\sum_{i=1}^8 \ln(k_i(u)!) \biggr\}}{\sum_{s \in D} \exp \biggl\{ -\sum_{i=1}^8 \ln(k_i(s)!) \biggr\}} =
    % $$
    % $$
    % = \dfrac{\exp \biggl\{ -\sum_{i=1}^8 f(k_i(u)) \biggr\}}{\sum_{s \in D} \exp \biggl\{ -\sum_{i=1}^8 f(k_i(s)) \biggr\}}
    % $$
    % Полученное выражение удобно с позиции того, что современные ЭВМ умеют вычислять функцию
    % $$
    % \varphi(x,i)=\dfrac{\exp\{x_i\}}{\sum_{j=1}^{n} \exp\{x_j\}}, \; x=(x_1,\ldots,x_n)
    % $$
    % За счет свойства
    % $$
    % \varphi(x,i)=\dfrac{\exp\{x_i\}}{\sum_{j=1}^{n} \exp\{x_j\}} = \dfrac{\exp\{x_i - C\}}{\sum_{j=1}^{n} \exp\{x_j - C\}}
    % , \text{ где } C=\max_{1\leq j \leq n} x_j
    % $$
    % удается избежать переполнения вещественного типа данных, связанного с экспонентой.
\subsection{Процедура проверки условной независимости
в трехмерном распределении Бернулли}\label{twos}

Пусть $(X,Y,Z)^T$ -- случайный вектор, имеющий трехмерное распределение
Бернулли, а
$
\begin{pmatrix}
        x_1 \\
        y_1 \\
        z_1
    \end{pmatrix},
    \begin{pmatrix}
        x_2 \\
        y_2 \\
        z_2
    \end{pmatrix}, \ldots,
    \begin{pmatrix}
        x_n \\
        y_n \\
        z_n
    \end{pmatrix}
$ реализация повторной выборки из распределения $(X,Y,Z)^T$.

Предложим процедуру проверки гипотезы $H: X \ci Y \mid Z$,
контролирующую вероятность ошибки первого рода на уровне $\alpha$:
\begin{itemize}
    \item Разобьем исходную выборку на две подвыборки:
    $$
    \begin{pmatrix}
            x_{i_1} \\
            y_{i_1} \\
            0
        \end{pmatrix},
        \begin{pmatrix}
            x_{i_2} \\
            y_{i_2} \\
            0
        \end{pmatrix}, \ldots,
        \begin{pmatrix}
            x_{i_{n_0}} \\
            y_{i_{n_0}} \\
            0
        \end{pmatrix}
    \text{ и }
    \begin{pmatrix}
            x_{j_1} \\
            y_{j_1} \\
            1
        \end{pmatrix},
        \begin{pmatrix}
            x_{j_2} \\
            y_{j_2} \\
            1
        \end{pmatrix}, \ldots,
        \begin{pmatrix}
            x_{j_{n_1}} \\
            y_{j_{n_1}} \\
            1
        \end{pmatrix}
    $$
    \item По наблюдениям 
    $
    \begin{pmatrix}
        x_{i_1} \\
        y_{i_1} 
    \end{pmatrix},
    \begin{pmatrix}
        x_{i_2} \\
        y_{i_2}
    \end{pmatrix}, \ldots,
    \begin{pmatrix}
        x_{i_{n_0}} \\
        y_{i_{n_0}}
    \end{pmatrix}
    $, которые являются реализацией повторной выборки
    из распределения $(X,Y)^T$ при условии $Z~=~0$,
тестом $\varphi^{\text{Independence}}$ (из \autoref{bivariate_umpu}) уровня $\gamma$
проверим гипотезу $H_0 : X$ и $Y$ независимы при условии $Z=0$. 
Если подвыборка не содержит наблюдений, то применяем тест 
$\varphi \equiv \gamma$.
\item По наблюдениям 
    $
    \begin{pmatrix}
        x_{j_1} \\
        y_{j_1} 
    \end{pmatrix},
    \begin{pmatrix}
        x_{j_2} \\
        y_{j_2}
    \end{pmatrix}, \ldots,
    \begin{pmatrix}
        x_{j_{n_1}} \\
        y_{j_{n_1}}
    \end{pmatrix}
    $, которые являются реализацией 
    повторной выборки из распределения 
    $(X,Y)^T$ при условии $Z~=~1$,
тестом $\varphi^{\text{Independence}}$ (из \autoref{bivariate_umpu}) уровня $\gamma$
проверим гипотезу $H_1 : X$ и $Y$ независимы при условии $Z=1$.
Если подвыборка не содержит наблюдений, то применяем тест 
$\varphi \equiv \gamma$.
\item 
Для проверки гипотезы $H: X \ci Y \mid Z$ используем тест
объединения-пересечения \cite{Roy1953}:
$$
\varphi^{\text{Subsamples}}=\begin{cases}
    1, \text{ наступило событие $A_0 \cup A_1$}\\
    0, \text{ иначе}
\end{cases}
$$
где $A_0 = \{\text{гипотеза $H_0$ отвергнута}\},\;
A_1 = \{\text{гипотеза $H_1$ отвергнута}\}$.
\end{itemize}
\begin{remark}
    Очевидно, что
события $A_0$ и $A_1$ независимые. Поэтому для контроля
$\Prb_{H}(\varphi^{\text{Subsamples}}=1)=\alpha$ достаточно положить
$\gamma = 1 - \sqrt{1-\alpha}$.
\end{remark}
\begin{remark}
    Использование в вышеприведенной процедуре теста\\
    $\varphi^{\text{Independence}}$ объясняется тем, что
    случайный вектор $(X,Y)^T$ при условии $Z~=~z$ имеет двумерное
    распределение Бернулли \cite{Dai2013}.
\end{remark}

% \newpage
% \subsection{Проверка условной независимости по подвыборкам из условных распределений}\label{twos}

% Приведем трактовку \autoref{cond_ind_def} для трехмерного
% распределения Бернулли. 
% \begin{definition}\label{new_def}
%     В трехмерном распределении Бернулли случайные величины $X$ и $Y$
%     условно независимы при условии $Z$, 
%     если выполнены следующие условия:
%     \begin{itemize}
%         \item $X$ и $Y$ независимы при условии $Z=0$
%         \item $X$ и $Y$  независимы при условии $Z=1$
%     \end{itemize}
% \end{definition}
% Используя \autoref{new_def}, сформулируем индивидуальные гипотезы:
% \begin{itemize}
%     \item $H_0 : X$ и $Y$ независимы при условии $Z=0$
%     \item $H_1 : X$ и $Y$ независимы при условии $Z=1$
% \end{itemize}
% Тогда гипотеза об условной независимости имеет вид 
% $H = H_0 \cap H_1$. Естественным образом, гипотезу 
% $H_0$ необходимо проверять по наблюдениям
% $(x_i,y_i,z_i)^T$, в которых $z_i=0$. Поскольку $(X,Y)^T$ при условии
% $Z=z$ имеет двумерное распределение Бернулли, то
% в качестве теста для $H_0$ можно использовать
% $\varphi_0 = \varphi^{\text{Independence}}_0$, 
% приведенный в \autoref{bivariate_umpu}. Аналогичные рассуждения 
% справедливы и для гипотезы $H_1$. Учитывая озвученные соображения,
% построим тест проверки гипотезы $H$, контролирующий вероятность
% ошибки первого рода на уровне $\alpha$.
% Пусть
% $$
% \begin{pmatrix}
%         X_1 \\
%         Y_1 \\
%         Z_1
%     \end{pmatrix},
%     \begin{pmatrix}
%         X_2 \\
%         Y_2 \\
%         Z_2
%     \end{pmatrix}, \ldots,
%     \begin{pmatrix}
%         X_n \\
%         Y_n \\
%         Z_n
%     \end{pmatrix}
% $$ повторная выборка из распределения случайного вектора $(X,Y,Z)^T$. Обозначим $\mathbf{Z}=(Z_1,\ldots,Z_n)$ и 
% $\mathbf{z}=(z_1,\ldots,z_n)$.
% Покажем, что в условном распределении при $\mathbf{Z}=\mathbf{z}$ наблюдения
% $$
% \Xi = 
% \begin{pmatrix}
%         X_1 \\
%         Y_1
%     \end{pmatrix},
%     \begin{pmatrix}
%         X_2 \\
%         Y_2
%     \end{pmatrix}, \ldots,
%     \begin{pmatrix}
%         X_n \\
%         Y_n
%     \end{pmatrix}
% $$
% являются независимыми. 

% \begin{lemma}\label{ci_for_samples}
%     $$
%     P(X_1=x_1,Y_1=y_1,\ldots,X_n=x_n,Y_n=y_n \mid \mathbf{Z}=\mathbf{z})=
%     $$
%     $$
%     =\prod_{i=1}^n P(X_i=x_i, Y_i=y_i \mid \mathbf{Z}=\mathbf{z})
%     $$
% \end{lemma}
% \begin{proof}
%     С одной стороны:
%     $$
%     P(X_1=x_1,Y_1=y_1,Z_1=z_1,\ldots,X_n=x_n,Y_n=y_n,Z_n=z_n)=
%     $$
%     $$
%     =P(\mathbf{Z}=\mathbf{z}) P(X_1=x_1,Y_1=y_1,\ldots,X_n=x_n,Y_n=y_n \mid \mathbf{Z}=\mathbf{z}) 
%     $$
%     С другой стороны:
%     $$
%     P(X_1=x_1,Y_1=y_1,Z_1=z_1,\ldots,X_n=x_n,Y_n=y_n,Z_n=z_n)=
%     $$
%     $$
%     =\prod_{i=1}^n P(X_i=x_i, Y_i=y_i, Z_i=z_i)=
%     $$
%     $$
%     =\prod_{i=1}^n P(X_i=x_i, Y_i=y_i \mid Z_i=z_i)P(Z_i=z_i)=
%     $$
%     $$
%     =P(\mathbf{Z}=\mathbf{z})\prod_{i=1}^n P(X_i=x_i, Y_i=y_i \mid \mathbf{Z}=\mathbf{z})
%     $$
% \end{proof}

% Пусть также $\mathbf{Z}=\mathbf{z}$ фиксированы. 
% Разобьем выборку $\Xi$ на две подвыборки $\Xi_0$ и $\Xi_1$, 
% такие что:
% $$
% \Xi_0=
% \begin{pmatrix}
%     X_{i_1} \\
%     Y_{i_1} \\
% \end{pmatrix},
% \begin{pmatrix}
%     X_{i_2} \\
%     Y_{i_2} \\
% \end{pmatrix}, \ldots,
% \begin{pmatrix}
%     X_{i_{n_0}} \\
%     Y_{i_{n_0}} \\
% \end{pmatrix} 
% $$
% где $Z_{i_k}=z_{i_k}=0$ для всех $i_k$ при $k=\overline{1,n_0}$ и $$
% \Xi_1=
% \begin{pmatrix}
%     X_{j_1} \\
%     Y_{j_1} \\
% \end{pmatrix},
% \begin{pmatrix}
%     X_{j_2} \\
%     Y_{j_2} \\
% \end{pmatrix}, \ldots,
% \begin{pmatrix}
%     X_{j_{n_1}} \\
%     Y_{j_{n_1}} \\
% \end{pmatrix} 
% $$
% где $Z_{j_k}=z_{j_k}=1$ для всех $j_k$ при $k=\overline{1,n_1}$. 
% Причем $n=n_0+n_1$. Отметим, что
% разбиение $\Xi = \Xi_0 \sqcup \Xi_1$ определяется лишь набором
% $\mathbf{Z}=\mathbf{z}$.

% Сформулируем следующую теорему.

% \begin{theorem}
%     Пусть $\mathbf{Z}=\mathbf{z}$ -- фиксированы. Тогда выборка $\Xi_0$ является повторной выборкой из распределения $(X,Y)^T$ при условии $Z=0$.
% \end{theorem}
% \begin{proof}
%     По \autoref{ci_for_samples}:
%     $$
%     P(X_1=x_1,Y_1=y_1,\ldots,X_n=x_n,Y_n=y_n \mid \mathbf{Z}=\mathbf{z})=
%     $$
%     $$
%     =\prod_{i=1}^n P(X_i=x_i, Y_i=y_i \mid \mathbf{Z}=\mathbf{z})
%     $$
%     Просуммировав обе части вышепредставленного равенства по всем возможным значениям 
%     $x_{j_1}, y_{j_1}, \ldots, x_{j_{n_1}},y_{j_{n_1}}$ получаем:
%     $$
%     P(X_{i_1}=x_{i_1},Y_{i_1}=y_{i_1},\ldots,X_{i_{n_0}}=x_{i_{n_0}},Y_{i_{n_0}}=y_{i_{n_0}} \mid \mathbf{Z}=\mathbf{z})=
%     $$
%     $$
%     =\prod_{k=1}^{n_0} P(X_{i_k}=x_{i_k}, Y_{i_k}=y_{i_k} \mid \mathbf{Z}=\mathbf{z})
%     $$
%     Значит $$
%     \Xi_0=
%     \begin{pmatrix}
%         X_{i_1} \\
%         Y_{i_1} \\
%     \end{pmatrix},
%     \begin{pmatrix}
%         X_{i_2} \\
%         Y_{i_2} \\
%     \end{pmatrix}, \ldots,
%     \begin{pmatrix}
%         X_{i_{n_0}} \\
%         Y_{i_{n_0}} \\
%     \end{pmatrix} 
%     $$
%     независимые наблюдения при условии $\mathbf{Z}=\mathbf{z}$.

%     Покажем, что $(X_{i_k},Y_{i_k})^T$ при условии $\mathbf{Z}=\mathbf{z}$ распределен также как и 
%     $(X,Y)^T$ при условии $Z=0$.
%     $$
%     P(X_{i_k}=x_{i_k}, Y_{i_k}=y_{i_k} \mid \mathbf{Z}=\mathbf{z}) =
%     P(X_{i_k}=x_{i_k}, Y_{i_k}=y_{i_k} \mid Z_{i_k}=z_{i_k}) = $$
%     $$=P(X_{i_k}=x_{i_k}, Y_{i_k}=y_{i_k} \mid Z_{i_{k}}=0)=
%     P(X=x_{i_k}, Y=y_{i_k} \mid Z=0)$$
% \end{proof}

% Аналогично показывается, что $\Xi_1$ является повторной выборкой из распределения $(X,Y)^T$ при условии $Z=1$.

% Сформулируем теорему.
% \begin{theorem}\label{main_theorem}
%     Пусть $\Xi_0$ и $\Xi_1$ -- подвыборки, полученные разбиением случайной выборки $\Xi$.
%     Пусть $\varphi_0$ и $\varphi_1$ -- рандомизированные тесты 
%     проверки гипотез $H_0$ и $H_1$ по повторным выборкам 
%     $\Xi_0$ и $\Xi_1$ соответственно.
%     Введем события:
%     $$A_0 = \{\text{отвергнуть гипотезу $H_0$ рандомизированным тестом $\varphi_0$}\}$$ 
%     $$A_1 = \{\text{отвергнуть гипотезу $H_1$ рандомизированным тестом $\varphi_1$}\}$$
%     Пусть $\varphi_0$ и $\varphi_1$ тесты уровня $\alpha_0$ и $\alpha_1$
%     при любом объеме наблюдений в подвыборках $\Xi_0$ и $\Xi_1$ соответственно, то есть:
%     $$P_{H_0\cap H_1,\mathbf{Z=z}}(A_0)=\alpha_0$$ 
%     $$P_{H_0\cap H_1,\mathbf{Z=z}}(A_1)=\alpha_1$$
%     Тогда $P_{H_0\cap H_1}(A_0 \cap A_1)= P_{H_0\cap H_1}(A_0) P_{H_0\cap H_1}(A_1)$.
% \end{theorem}
% \begin{proof}
%     Пусть $\mathbf{Z}=\mathbf{z}$ фиксировано. 
%     Обозначим через
%     $$T_0=(X_{i_1},Y_{i_1},\ldots,X_{i_{n_0}},Y_{i_{n_0}})^T, \;
%     T_1=(X_{j_1},Y_{j_1},\ldots,X_{j_{n_1}},Y_{j_{n_1}})^T$$
%     случайные векторы наблюдений, используемые в тестах $\varphi_0$ и 
%     $\varphi_1$ соответственно.
%     Распишем следующую вероятность
%     $$
%     P_{H_0\cap H_1,\mathbf{Z=z}}(A_0 \cap A_1)=
%     $$
%     $$
%     =\sum_{t_0}\sum_{t_1} P_{H_0\cap H_1,\mathbf{Z=z}}(A_0 \cap A_1 \mid T_0=t_0, T_1=t_1)P_{H_0\cap H_1,\mathbf{Z=z}}(T_0=t_0, T_1=t_1)
%     $$
%     Отметим, что $P_{H_0\cap H_1,\mathbf{Z=z}}(A_0 \cap A_1 \mid T_0=t_0, T_1=t_1)=\varphi_0(t_0)\varphi_1(t_1)$, поскольку
%     для того, чтобы при известных значениях статистик $t_0$ и $t_1$ отвергнуть гипотезы $H_0$ и $H_1$, в рандомизированном тесте нужно провести два испытания с вероятностью успеха
%     $\varphi_0(t_0)$ и $\varphi_1(t_1)$. Постулируется, что такие испытания независимые. Тогда:
%     $$
%     P_{H_0\cap H_1,\mathbf{Z=z}}(A_0 \cap A_1)=\sum_{t_0}\sum_{t_1} \varphi_0(t_0) \varphi_1(t_1) P_{H_0\cap H_1,\mathbf{Z=z}}(T_0=t_0, T_1=t_1)=
%     $$
%     $$
%     =\sum_{t_0}\sum_{t_1} \varphi_0(t_0) \varphi_1(t_1) P_{H_0\cap H_1,\mathbf{Z=z}}(T_0=t_0)P_{H_0\cap H_1,\mathbf{Z=z}}(T_1=t_1)=
%     $$
%     $$
%     =\sum_{t_0}\left[ \varphi_0(t_0) P_{H_0\cap H_1,\mathbf{Z=z}}(T_0=t_0) \left(\sum_{t_1}\varphi_1(t_1) P_{H_0\cap H_1,\mathbf{Z=z}}(T_1=t_1)\right)\right]=
%     $$
%     $$
%     = \sum_{t_0} \varphi_0(t_0) P_{H_0\cap H_1,\mathbf{Z=z}}(T_0=t_0) \alpha_1 
%     =\alpha_0 \alpha_1
%     $$
%     По формуле полной вероятности:
%     $$
%     P_{H_0\cap H_1}(A_0) = \sum_{\mathbf{z}} P_{H_0\cap H_1,\mathbf{Z=z}}(A_0) P_{H_0\cap H_1}(\mathbf{Z}=\mathbf{z})
%     = \sum_{\mathbf{z}} \alpha_0 P_{H_0\cap H_1}(\mathbf{Z}=\mathbf{z})=\alpha_0
%     $$
%     Аналогично, $P_{H_0\cap H_1}(A_1)=\alpha_1$. Также по формуле полной вероятности:
%     $$
%     P_{H_0\cap H_1}(A_0 \cap A_1) = \sum_{\mathbf{z}} P_{H_0\cap H_1,\mathbf{Z=z}}(A_0 \cap A_1) P_{H_0\cap H_1}(\mathbf{Z}=\mathbf{z})=
%     $$
%     $$
%     = \sum_{\mathbf{z}} \alpha_0 \alpha_1 P_{H_0\cap H_1}(\mathbf{Z}=\mathbf{z})
%     = \alpha_0 \alpha_1
%     $$
%     Таким образом, $P_{H_0\cap H_1}(A_0 \cap A_1)=P_{H_0\cap H_1}(A_0) P_{H_0\cap H_1}(A_1)=\alpha_0 \alpha_1$.
% \end{proof}
% Применим \autoref{main_theorem} для проверки условной независимости в трехмерном
% распределении Бернулли.
% Положим индивидуальные гипотезы:
% \begin{itemize}
%     \item $H_0 : X$ и $Y$ независимы при условии $Z=0$
%     \item $H_1 : X$ и $Y$ независимы при условии $Z=1$
% \end{itemize}
% Для проверки гипотез $H_0$ и $H_1$ будем использовать тесты
% $\varphi_0 = \varphi^{\text{Independence}}_0$ и 
% $\varphi_1 = \varphi^{\text{Independence}}_1$ уровня $\alpha_0$ и $\alpha_1$ 
% по повторным выборкам
% $\Xi_0$ и $\Xi_1$ соответственно.
% Тогда гипотеза
% об условной независимости имеет вид $H = H_0 \cap H_1$ и тест проверки условной независимости можно определить как:
% $$
% \varphi^{\text{Subsamples}}=\begin{cases}
%     1, \text{ наступило событие $A_0 \cup A_1$}\\
%     0, \text{ иначе}
% \end{cases}
% $$
% Пусть далее $\alpha_0 = \alpha_1 = \gamma$.
% Тогда $$P_{H_0\cap H_1}(\varphi^{\text{Subsamples}}=1)=P_{H_0\cap H_1}(A_0 \cup A_1)=$$ 
% $$ = P_{H_0\cap H_1}(A_0) + P_{H_0\cap H_1}(A_1) - 
% P_{H_0\cap H_1}(A_0 \cap A_1) = 2\gamma - \gamma^2$$
% Нетрудно проверить, что для контроля $P_{H_0\cap H_1}(\varphi^{\text{Subsamples}}=1)=\alpha$ достаточно положить
% уровень значимости $\gamma = 1 - \sqrt{1-\alpha}$ на
% тестах проверки индивидуальных гипотез $H_0$ и $H_1$.

% Покажем, что тест $\varphi^{\text{Subsamples}}$ является несмещенным.
% \begin{theorem}\label{unbias}
%     Тест $\varphi^{\text{Subsamples}}$
%     уровня $\alpha$
%     проверки гипотезы 
%     $H: X \ci Y \mid Z$ является несмещенным.
% \end{theorem}
% \begin{proof}
%     Положим $\Theta_{H}=\{\theta: p_{000}p_{110}=p_{010}p_{100}
%     \text{ и } p_{001}p_{111}=p_{011}p_{101}\}$, 
%     $\Theta_{K}=\{\theta: p_{000}p_{110}\neq p_{010}p_{100}
%     \text{ или } p_{001}p_{111}\neq p_{011}p_{101}\}$.

%     Отметим, что $\varphi_0 = \varphi^{\text{Independence}}_0$ и
%     $\varphi_1 = \varphi^{\text{Independence}}_1$ -- несмещенные тесты
%     проверки гипотез  $H_0$ и $H_1$ уровня $\gamma$
%     соответственно. Причем, при истинности гипотез $H_0$
%     и $H_1$ на тестах $\varphi_0$ и $\varphi_1$ вероятность ошибки
%     первого рода в точности равна $\gamma$.
%     Поэтому,
%     $E_{\theta}(\varphi_0) = P_\theta(A_0)\geq \gamma$ и $E_{\theta}(\varphi_1) = P_\theta(A_1)\geq \gamma$ для любого
%     $\theta \in \Theta_{H} \cup \Theta_{K}$.

%     Пусть $\theta \in \Theta_{H}$, тогда
%     $E_{\theta}(\varphi^{\text{Subsamples}}) =
%     P_{\theta}(\varphi^{\text{Subsamples}}=1) = \alpha$ поскольку
%     $\varphi^{\text{Subsamples}}$ тест уровня $\alpha$.

%     Пусть $\theta \in \Theta_{K}$.
%     Как и в доказательстве \autoref{main_theorem} можно показать, что
%     $P_{\theta,\mathbf{Z}=\mathbf{z}}(A_0 \cap A_1)=
%     P_{\theta,\mathbf{Z}=\mathbf{z}}(A_0)
%     P_{\theta,\mathbf{Z}=\mathbf{z}}(A_1)$.
%     Следовательно: 
%     $$P_{\theta,\mathbf{Z}=\mathbf{z}}(A_0 \cup A_1) 
%     = P_{\theta,\mathbf{Z}=\mathbf{z}}(A_0) +
%     P_{\theta,\mathbf{Z}=\mathbf{z}}(A_1) - 
%     P_{\theta,\mathbf{Z}=\mathbf{z}}(A_0)
%     P_{\theta,\mathbf{Z}=\mathbf{z}}(A_1) 
%     \geq \gamma + \gamma - \gamma^2 = \alpha
%     $$
%     Тогда:
%     $$
%     E_{\theta}(\varphi^{\text{Subsamples}}) =
%     P_{\theta}(\varphi^{\text{Subsamples}}=1)=
%     P_{\theta}(A_0 \cup A_1)=
%     $$
%     $$
%     =\sum_{\mathbf{z}} P_{\theta,\mathbf{Z}=\mathbf{z}}(A_0 \cup A_1)P_{\theta}(\mathbf{Z}=\mathbf{z}) \geq
%     \sum_{\mathbf{z}} \alpha P_{\theta}(\mathbf{Z}=\mathbf{z}) = \alpha
%     $$
    
% \end{proof}

% \section{Тест проверки условной независимости в трехмерном нормальном распределении}
Пусть
$$
\begin{pmatrix}
    X_1\\
    Y_1 \\
    Z_1
\end{pmatrix}, \begin{pmatrix}
    X_2\\
    Y_2 \\
    Z_2
\end{pmatrix}, \ldots, 
\begin{pmatrix}
    X_n \\
    Y_n \\
    Z_n
\end{pmatrix}
$$
повторная выборка из распределения $(X,Y,Z)^T$ с трехмерным нормальным распределением $\mathcal{N}(\mu, \Sigma)$,
где $\mu$ -- вектор математических ожиданий, а $\Sigma$ -- ковариационная матрица:
$$
\mu = \begin{pmatrix}
    EX \\ EY \\ EZ
\end{pmatrix}, 
\;
\Sigma = \begin{pmatrix}
    \sigma_{XX} & \sigma_{XY} & \sigma_{XZ} \\
    \sigma_{YX} & \sigma_{YY} & \sigma_{YZ} \\
    \sigma_{ZX} & \sigma_{ZY} & \sigma_{ZZ} 
\end{pmatrix}
$$
а также $\sigma_{XY}=E[(X-EX)(Y-EY)]$.
Для матрицы $\Sigma$ элементы обратной матрицы $\Sigma^{-1}$ будем обозначать:
$$
\Sigma^{-1} =
\begin{pmatrix}
    \sigma^{XX} & \sigma^{XY} & \sigma^{XZ} \\
    \sigma^{YX} & \sigma^{YY} & \sigma^{YZ} \\
    \sigma^{ZX} & \sigma^{ZY} & \sigma^{ZZ} 
\end{pmatrix}
$$

\begin{definition}
    В трехмерном нормальном распределении частным коэффициентом корреляции Пирсона называется:
    $$
    \rho^{XY\cdot Z} = \dfrac{\sigma^{XY}}{\sqrt{\sigma^{XX} \sigma^{YY}}}
    $$
\end{definition}
Известно, что в трехмерном нормальном распределении частный коэффициент корреляции совпадает с условным коэффициентом корреляции 
и выполнено:
$$
X \ci Y \mid Z \text{ тогда и только тогда, когда } \rho^{XY\cdot Z}=0
$$
\begin{definition}
    Выборочной ковариационной матрицей называется матрица $S$, такая что:
    $$S= 
    \begin{pmatrix}
        s_{XX} & s_{XY} & s_{XZ}\\
        s_{YX} & s_{YY} & s_{YZ}\\
        s_{ZX} & s_{ZY} & s_{ZZ}
    \end{pmatrix}
    $$
    где $$
    s_{XY} = \dfrac{1}{n-1} \sum_{i=1}^{n} (X_i-\overline{X})(Y_i - \overline{Y}), \; \overline{X} = \dfrac{1}{n} \sum_{i=1}^{n} X_i
    $$
\end{definition}
Для матрицы $S$ элементы обратной матрицы $S^{-1}$ будем обозначать:
$$
S^{-1}=\begin{pmatrix}
    s^{XX} & s^{XY} & s^{XZ}\\
    s^{YX} & s^{YY} & s^{YZ}\\
    s^{ZX} & s^{ZY} & s^{ZZ}
\end{pmatrix}
$$
\begin{definition}
    Выборочным частным коэффициентом корреляции Пирсона называется:
    $$r^{XY\cdot Z}=\dfrac{s^{XY}}{\sqrt{s^{XX} s^{YY}}}$$
\end{definition}

Известно, что при истинности гипотезы $\rho^{XY \cdot Z}=0$ статистика
$$
T=\sqrt{n-3} \dfrac{r^{XY \cdot Z}}{\sqrt{1-(r^{XY \cdot Z})^2}}
$$
имеет распределение Стьюдента с $n-3$ степенями свободы.
Тогда тест уровня $\alpha$ проверки гипотезы $H: \rho^{XY\cdot Z}=0$ против альтернативы $K: \rho^{XY\cdot Z} \neq 0$ имеет вид:
$$
\varphi(t) = \begin{cases}
    1, \; t<C_1 \text{ или } t>C_2 \\ 
    0, \; C_1 \leq t \leq C_2
\end{cases}
$$
где константы $C_1$ и $C_2$, удовлетворяющие уравнениям $P(T<C_1)~=~\alpha/2$ и $P(T>C_2)=1-\alpha/2$,
берутся из таблиц квантилей распределения Стьюдента с $n-3$ степенями свободы.




% % библиография
\newpage
% \nocite{*}
\addcontentsline{toc}{section}{Список использованной литературы}
\printbibliography
\end{document}
