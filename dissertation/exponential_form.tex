Трехмерное распределение Бернулли можно представить в экспоненциальном виде:
$$P(X=x,Y=y,Z=z)=p_{000}^{(1-x)(1-y)(1-z)} \ldots p_{111}^{x y z}$$

Докажем лемму касательно вида $P(X=x,Y=y,Z=z)$.

\begin{lemma}\label{factorization}
    $$
    P(X=x,Y=y,Z=z)= \exp \Biggl\{ \ln p_{000} \Biggr\}
        \exp \Biggl\{  xyz \ln  \left(\dfrac{p_{001}p_{111}p_{010}p_{100}}{p_{011}p_{101}p_{000}p_{110}}\right) +$$
    $$ +
        x \ln\left(\dfrac{p_{100}}{p_{000}}\right) +  y \ln\left(\dfrac{p_{010}}{p_{000}}\right) +
        z \ln\left(\dfrac{p_{001}}{p_{000}}\right) +
    $$
    $$
        + xy \ln \left(\dfrac{p_{000}p_{110}}{p_{010}p_{100}}\right) +
        xz \ln \left(\dfrac{p_{000}p_{101}}{p_{001}p_{100}}\right) +
        yz \ln \left(\dfrac{p_{000}p_{011}}{p_{001}p_{010}}\right) \Biggr\}
    $$
\end{lemma}

\begin{proof}
    $$
    P(X=x,Y=y,Z=z) = p_{000}^{(1-x)(1-y)(1-z)} \ldots p_{111}^{x y z} =
    $$
    $$
        =\exp \Biggl\{ (1-x)(1-y)(1-z) \ln p_{000} +
        (1-x)(1-y)z \ln p_{001}+
    $$
    $$
        + (1-x)y(1-z) \ln p_{010} + (1-x)y z \ln p_{011} +  x(1-y)(1-z) \ln p_{100} +
    $$
    $$
        +   x(1-y) z \ln p_{101}
        +   x y (1-z) \ln p_{110} +   x y z \ln p_{111} \Biggr\} =
    $$
    $$
        =\exp \Biggl\{   ( 1 - y -  x +  x y
        -  z +  y z +  x z -  x y z ) \ln p_{000} +
    $$
    $$
        +    (z -  y z -  x z +  x y z) \ln p_{001}  +
          (y -  y z -  x y +  x y z)  \ln p_{010} +
    $$
    $$
        +    (y z -  x y z ) \ln p_{011} +
           (x -  x z -  x y +  x y z ) \ln p_{100} +
    $$
    $$
        +   (x z -  x y z ) \ln p_{101} +   (x y -  x y z) \ln p_{110} +
          x y z \ln p_{111} \Biggr\}=
    $$
    $$
        = \exp \Biggl\{ \ln p_{000} \Biggr\}
        \exp \Biggl\{  xyz \ln  \left(\dfrac{p_{001}p_{111}p_{010}p_{100}}{p_{011}p_{101}p_{000}p_{110}}\right) +$$
    $$ +
        x \ln\left(\dfrac{p_{100}}{p_{000}}\right) +  y \ln\left(\dfrac{p_{010}}{p_{000}}\right) +
        z \ln\left(\dfrac{p_{001}}{p_{000}}\right) +
    $$
    $$
        + xy \ln \left(\dfrac{p_{000}p_{110}}{p_{010}p_{100}}\right) +
        xz \ln \left(\dfrac{p_{000}p_{101}}{p_{001}p_{100}}\right) +
        yz \ln \left(\dfrac{p_{000}p_{011}}{p_{001}p_{010}}\right) \Biggr\}
    $$
\end{proof}

Сформулируем теорему касательно параметра трехмерного распределения Бернулли, связанного с условной независимостью.

\begin{theorem}
    Пусть $\theta = \ln  \left(\dfrac{p_{001}p_{111}p_{010}p_{100}}{p_{011}p_{101}p_{000}p_{110}}\right)$.
    Если выполнено одно из условий:
    \begin{itemize}
        \item $X \ci Y \mid Z$
        \item $X \ci Z \mid Y$
        \item $Y \ci Z \mid X$
    \end{itemize}
    то параметр $\theta$ принимает значение $0$.
    \end{theorem}
    
    \begin{proof}
        Результаты теоремы $\ref{thm1}$ можно обобщить следующим образом:
        $$
        X \ci Z \mid Y \Leftrightarrow p_{000}p_{101}=p_{001}p_{100} \text{ и } p_{010}p_{111}=p_{011}p_{110}
        $$
        $$
        Y \ci Z \mid X \Leftrightarrow p_{000}p_{011}=p_{001}p_{010} \text { и } p_{100}p_{111}=p_{101}p_{110}
        $$
        \begin{enumerate}
            \item Пусть $X \ci Y \mid Z$, тогда по теореме \ref{thm1} выполнено:
            $p_{000}p_{110}=p_{010}p_{100}$ и  $p_{001}p_{111}=p_{011}p_{101}$. Отсюда следует, что
            $\theta=\ln(1)=0$.
            \item Пусть $X \ci Z \mid Y$, тогда из вышеприведенных соображений
            $p_{000}p_{101}=p_{001}p_{100}$ и $p_{010}p_{111}=p_{011}p_{110}$. Отсюда следует, что
            $\theta=\ln(1)=0$.
            \item Пусть $Y \ci Z \mid X$, тогда из вышеприведенных соображений
            $p_{000}p_{011}=p_{001}p_{010}$ и $p_{100}p_{111}=p_{101}p_{110}$. Отсюда следует, что
            $\theta=\ln(1)=0$.
        \end{enumerate}
    \end{proof}
