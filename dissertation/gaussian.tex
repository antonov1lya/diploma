\section{Тест проверки условной независимости в трехмерном нормальном распределении}
Пусть
$$
\begin{pmatrix}
    X_1\\
    Y_1 \\
    Z_1
\end{pmatrix}, \begin{pmatrix}
    X_2\\
    Y_2 \\
    Z_2
\end{pmatrix}, \ldots, 
\begin{pmatrix}
    X_n \\
    Y_n \\
    Z_n
\end{pmatrix}
$$
повторная выборка из распределения $(X,Y,Z)^T$ с трехмерным нормальным распределением $N(\mu, \Sigma)$,
где $\mu$ -- вектор математических ожиданий, а $\Sigma$ -- ковариационная матрица:
$$
\mu = \begin{pmatrix}
    EX \\ EY \\ EZ
\end{pmatrix}, 
\;
\Sigma = \begin{pmatrix}
    \sigma_{XX} & \sigma_{XY} & \sigma_{XZ} \\
    \sigma_{YX} & \sigma_{YY} & \sigma_{YZ} \\
    \sigma_{ZX} & \sigma_{ZY} & \sigma_{ZZ} 
\end{pmatrix}
$$
а также $\sigma_{XY}=E((X-EX)(Y-EY))$.
\begin{definition}
    В трехмерном нормальном распределении частным коэффициентом корреляции Пирсона называется:
    $$
    \rho^{XY\cdot Z} = \dfrac{\rho_{XY}-\rho_{XZ}\rho_{YZ}}{\sqrt{1-\rho_{XY}^2}\sqrt{1-\rho_{YZ}^2}}
    $$
\end{definition}
Известно, что в трехмерном нормальном распределении частный коэффициент корреляции совпадает с условным коэффициентом корреляции 
и выполнено:
$$
X \ci Y \mid Z \text{ тогда и только тогда, когда } \rho^{XY\cdot Z}=0
$$

\begin{definition}
    Выборочным коэффициентом корреляции Пирсона называется:
    $$
    r_{XY}=\dfrac{\sum_{i=1}^{n} (X_i-\overline{X})(Y_i - \overline{Y})}{
        \sqrt{\sum_{i=1}^{n} (X_i-\overline{X})^2}
        \sqrt{\sum_{i=1}^{n} (Y_i-\overline{Y})^2}
    }
    $$
    где
    $$
    \overline{X} = \dfrac{1}{n} \sum_{i=1}^{n} X_i
    $$
\end{definition}

\begin{definition}
    Выборочным частным коэффициентом корреляции Пирсона называется:
    $$r^{XY\cdot Z}=\dfrac{r_{XY}-r_{XZ}r_{YZ}}{\sqrt{1-r_{XY}^2}\sqrt{1-r_{YZ}^2}}$$
\end{definition}

Известно, что при истинности гипотезы $\rho^{XY \cdot Z}=0$ статистика
$$
T=\sqrt{n-3} \dfrac{r^{XY \cdot Z}}{\sqrt{1-(r^{XY \cdot Z})^2}}
$$
имеет распределение Стьюдента с $n-3$ степенями свободы.
Тогда тест уровня $\alpha$ проверки гипотезы $H: \rho^{XY\cdot Z}=0$ против альтернативы $K: \rho^{XY\cdot Z} \neq 0$ имеет вид:
$$
\varphi(t) = \begin{cases}
    1, \; t<C_1 \text{ или } t>C_2 \\ 
    0, \; C_1 \leq t \leq C_2
\end{cases}
$$
где константы $C_1$ и $C_2$, удовлетворяющие уравнениям $P(T<C_1)~=~\alpha/2$ и $P(T>C_2)=1-\alpha/2$,
берутся из таблиц квантилей распределения Стьюдента с $n-3$ степенями свободы.
