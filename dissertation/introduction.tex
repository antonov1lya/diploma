\section*{Введение}
\addcontentsline{toc}{section}{Введение}

\textbf{Обзор по теме исследования и актуальность} \quad
В современных задачах биоинформатики, информационного поиска,
обработки речи и изображений возникает необходимость изучения
взаимосвязей между большим количеством случайных величин.
Методологию для решения такой проблемы предоставляют графические модели. 
Графической моделью \cite{Jordan2004} называется 
семейство вероятностных распределений, определенное в терминах
ориентированного или неориентированного графа. В этом графе вершины 
соответствуют случайным величинам, 
а ребра отображают некоторые условные зависимости между случайными величинами. 
Основным назначением графических моделей является создание
аппарата, упрощающего вычисление
совместных, маргинальных и условных вероятностей.

Наиболее известной графической моделью является 
гауссовская графическая модель \cite{Anderson2003}, в которой рассматриваемые
случайные величины имеют многомерное нормальное распределение. Процедуры
идентификации гауссовской графической модели по наблюдениям
приводятся в работах \cite{Drton2004, Drton2007}. Однако, 
эти процедуры оказываются неустойчивыми к отклонению от многомерного
нормального распределения.
В частности, при таком отклонении тесты проверки индивидуальных гипотез 
не контролируют уровень значимости. 
Кроме того, многомерное нормальное распределение не всегда является
адекватной моделью для описания реальных данных. Поэтому проблема
построения устойчивой графической модели является актуальной. 

Направление построения устойчивой графической модели
для произвольного случайного вектора $(X_1,\ldots,X_N)^T$
целесообразно связать с построением графической модели для
индикаторных случайных величин $(I_1,\ldots,I_N)^T$, где 
$I_i=I(a_i<X_i<b_i)$. 
Возможной графической моделью в таком случае можно назвать
0-1 модель \cite{low_order1}.
В ней ребро между вершинами, которые соответствуют
случайным величинам $I_i$ и $I_j$,
проводится, если $I_i$ и $I_j$ условно зависимы при условии $I_k$
для всех $k\in \{1,\ldots,N\} \setminus \{i,j\}$.
Так как совместное распределение индикаторных случайных величинам
описывается многомерным распределением Бернулли 
\cite{Dai2013, Teugels1990}, то
для идентификации такой 
графической модели
по наблюдениям требуется
теория проверки условной независимости в трехмерном распределении
Бернулли. Построению требуемой теории посвящена
данная работа.

% Естественным обобщением многомерного нормального распределения
% является эллиптическое распределение \cite{Anderson2003}.
% Стоит отметить, что класс многомерных нормальных распределений 
% содержится в классе эллиптических распределений.
% Одним из параметров эллиптического распределения выступает $S$ -- матрица 
% формы (shape matrix), которая при существовании вторых моментов
% компонент эллиптического случайного вектора пропорциональна ковариационной матрице.
% В работе \cite{Vogel2011} для построения эллиптической 
% графической модели, 
% как устойчивого аналога гауссовской графической модели, вводится $K$ -- матрица обобщенных частных корреляций -- как функция от матрицы $S$.
% Ноль в матрице обобщенных частных корреляций
% обозначает условную некоррелированность пары случайных величин при условии
% остальных случайных величин. То есть в отличии от гауссовской
% графической модели, в эллиптической графической модели отсутствие
% ребра обозначает не условную независимость, а условную некоррелированность.
% В работе \cite{Vogel2011} рассматривают 
% $\hat K$ -- оценку матрицы $K$ --
% как функцию от $\hat S$ -- оценки матрицы $S$.
% Для $\hat S$ вводятся некоторые предположения,
% при выполнении которых вектор $Vec(\hat K)$, составленный из элементов $\hat K$, сходится по распределению
% к многомерному нормальному распределению с некоторыми параметрами. 
% Затем приводятся примеры таких оценок $\hat S$, 
% для которых эти предположения выполнены 
% при условии существовании некоторых моментов нормы
% разности эллиптического случайного вектора и его вектора средних.
% Таким образом, наличие оценок для $K$ и их асимптотических распределений приводит к статистической
% теории идентификации эллиптической графической модели.

% Другое направление построения устойчивой графической модели 
% можно связать с переходом от рассмотрения случайных величин $X_i$
% к рассмотрению индикаторных случайных величин $I_i=I(a_i<X_i<b_i)$.
% Условные зависимости между такими случайными величинами естественно
% приводят к графической модели.
% Совместное распределение индикаторных случайных величин описывается
% многомерным распределением Бернулли \cite{Dai2013, Teugels1990}.
% Настоящая выпускная квалификационная работа посвящена проверке условной
% независимости в трехмерном распределении Бернулли. В частности, эта
% теория может быть использована для задачи идентификации графической модели
% с попарным марковским свойством в трехмерном распределении Бернулли.
% Предполагается, что рассмотрение трехмерного случая позволит
% понять перспективность исследований общего многомерного случая.

\textbf{Постановка задачи} \quad
Задачей настоящей выпускной квалификационной работы 
является построение тестов уровня $\alpha$ 
проверки гипотезы \\ $H: X \ci Y \mid Z$
по наблюдениям
$
\begin{pmatrix}
    x_1 \\
    y_1 \\
    z_1
\end{pmatrix},
\begin{pmatrix}
    x_2 \\
    y_2 \\
    z_2
\end{pmatrix}, \ldots,
\begin{pmatrix}
    x_n \\
    y_n \\
    z_n
\end{pmatrix}
$ над случайным вектором $(X,Y,Z)^T$ с
трехмерным распределением Бернулли, а также
анализ свойств этих тестов при отклонении от условной независимости.
