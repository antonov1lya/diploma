\section{Частный коэффициент корреляции Пирсона}
Для удобства введем следующие обозначения: $$p_{x**}=P(X=x), \; p_{*y*}=P(Y=y), \; p_{**z}=P(Z=z)$$
$$p_{xy*}=P(X=x, Y=y), \; p_{x*z}=P(X=x, Z=z), \; p_{*yz}=P(Y=y, Z=z)$$
Символом $\Sigma$ будем обозначать ковариационную матрицу:
$$\Sigma =
    \begin{pmatrix}
        \sigma_{11} & \sigma_{12} & \sigma_{13} \\
        \sigma_{21} & \sigma_{22} & \sigma_{23} \\
        \sigma_{31} & \sigma_{32} & \sigma_{33}
    \end{pmatrix}
$$
Легко проверить, что $\sigma_{11}=D(X) = p_{1**}(1-p_{1**})$.

\begin{lemma}
    $$\sigma_{12}=\text{Cov}(X,Y)=p_{11*}-p_{1**}p_{*1*}$$
\end{lemma}

\begin{proof}
    Воспользуемся формулой $\text{Cov}(X,Y)=E(X Y)-$ $-E(X)E(Y)$.
    $$E(X Y) = 1 \cdot p_{11*} + 0 \cdot (p_{00*} + p_{01*} + p_{10*})=p_{11*}$$
    $$EX = 1 \cdot p_{1**} + 0 \cdot p_{0**}=p_{1**}$$
    $$ EY = 1 \cdot p_{*1*} + 0 \cdot p_{*0*} = p_{*1*}$$
    Таким образом, $\text{Cov}(X,Y)=p_{11*}-p_{1**}p_{*1*}$.
\end{proof}

Частный коэффициент корреляции Пирсона определяется через элементы обратной ковариационной матрицы $\Sigma^{-1}$:
$$
    \Sigma^{-1}=\begin{pmatrix}
        \sigma^{11} & \sigma^{12} & \sigma^{13} \\
        \sigma^{21} & \sigma^{22} & \sigma^{23} \\
        \sigma^{31} & \sigma^{32} & \sigma^{33}
    \end{pmatrix}
$$
Известно, что элемент $\sigma^{12}$ матрицы $\Sigma^{-1}$ выражается через соотношение
$\sigma^{12}=\dfrac{(-1)^{2+1}}{\text{det} (\Sigma)}M_{21}$, где
$
    M_{21}=\text{det}
    \begin{pmatrix}
        \sigma_{12} & \sigma_{13} \\
        \sigma_{32} & \sigma_{33}
    \end{pmatrix}
$.

\begin{lemma}\label{partial_cov}
    $$M_{21} = p_{**0}(p_{001}p_{111}-p_{011}p_{101}) + p_{**1} (p_{000}p_{110}-p_{010}p_{100})$$
\end{lemma}
\begin{proof}
    $$ M_{21}= \text{det}
        \begin{pmatrix}
            \sigma_{12} & \sigma_{13} \\
            \sigma_{23} & \sigma_{33}
        \end{pmatrix}
        = (p_{11*}-p_{1**}p_{*1*}) p_{**1}(1-p_{**1})-
    $$
    $$
        -(p_{1*1}-p_{1**}p_{**1})(p_{*11}-p_{*1*}p_{**1})=
    $$
    $$
        =p_{11*}p_{**1} - p_{11*}p_{**1}p_{**1} - p_{1**}p_{*1*}p_{**1} + p_{1**}p_{*1*}p_{**1}p_{**1}-
    $$
    $$
        -p_{1*1}p_{*11}+p_{1*1}p_{*1*}p_{**1}+p_{1**}p_{**1}p_{*11}-p_{1**}p_{**1}p_{*1*}p_{**1}=
    $$
    Заметим, что четвертое и восьмое слагаемые сокращаются. Распишем первое слагаемое как сумму вероятностей:
    $$
        =p_{111}p_{**1}+p_{110}p_{**1} - p_{11*}p_{**1}p_{**1} - p_{1**}p_{*1*}p_{**1} -
    $$
    $$
        -p_{1*1}p_{*11}+p_{1*1}p_{*1*}p_{**1}+p_{1**}p_{**1}p_{*11}=
    $$
    Осуществим перегруппировку слагаемых:
    $$
        =(p_{111}p_{**1}-p_{1*1}p_{*11})+p_{**1}(p_{110}-p_{11*}p_{**1} - p_{1**}p_{*1*} + p_{1*1}p_{*1*} + p_{1**}p_{*11})
    $$
    Преобразуем выражения для отдельных слагаемых.
    Заметим, что:
    $$
        p_{110}-p_{11*}p_{**1}=p_{110}-p_{110}p_{**1}-p_{111}p_{**1}=
    $$
    $$
        =p_{110}(1-p_{**1})-p_{111}p_{**1}=p_{110}p_{**0}-p_{111}p_{**1}
    $$
    Также заметим, что:
    $$
        -p_{1**}p_{*1*} + p_{1*1}p_{*1*} + p_{1**}p_{*11}=
    $$
    $$
        =-(p_{1*0}+p_{1*1})(p_{*10}+p_{*11})+p_{1*1}(p_{*10}+p_{*11}) + (p_{1*0}+p_{1*1})p_{*11}=
    $$
    $$
        =-p_{1*0}p_{*10}-p_{1*0}p_{*11}-p_{1*1}p_{*10}-p_{1*1}p_{*11}+
    $$
    $$
        +p_{1*1}p_{*10}+p_{1*1}p_{*11}+p_{1*0}p_{*11}+p_{1*1}p_{*11}=
    $$
    $$
        =-p_{1*0}p_{*10}+p_{1*1}p_{*11}
    $$
    Запишем выражение для $M_{21}$ с преобразованными слагаемыми:
    $$
        M_{21}=(p_{111}p_{**1}-p_{1*1}p_{*11})+p_{**1}((p_{110}p_{**0}-p_{1*0}p_{*10})-(p_{111}p_{**1}-p_{1*1}p_{*11}))=
    $$
    $$
        =(p_{111}p_{**1}-p_{1*1}p_{*11})+p_{**1}(p_{110}p_{**0}-p_{1*0}p_{*10})-p_{**1}(p_{111}p_{**1}-p_{1*1}p_{*11})=
    $$
    $$
        =(1-p_{**1})(p_{111}p_{**1}-p_{1*1}p_{*11})+p_{**1}(p_{110}p_{**0}-p_{1*0}p_{*10})=
    $$
    $$
        =p_{**0}(p_{111}p_{**1}-p_{1*1}p_{*11})+p_{**1}(p_{110}p_{**0}-p_{1*0}p_{*10})
    $$
    Снова преобразуем отдельные слагаемые.
    $$
        p_{111}p_{**1}-p_{1*1}p_{*11} = p_{111}(p_{001}+p_{011}+p_{101}+p_{111})-
        (p_{101}+p_{111})(p_{011}+p_{111})=
    $$
    $$
        = p_{111}p_{001}+p_{111}p_{011}+p_{111}p_{101}+p_{111}p_{111}
        - p_{101}p_{011}-p_{101}p_{111}-p_{111}p_{011}-p_{111}p_{111}=
    $$
    $$
        = p_{001}p_{111}-p_{011}p_{101}
    $$
    Аналогично преобразуем выражение:
    $$
        p_{110}p_{**0}-p_{1*0}p_{*10}=
        p_{110}(p_{000}+p_{010}+p_{100}+p_{110})-(p_{100}+p_{110})(p_{010}+p_{110})=
    $$
    $$
        =p_{110}p_{000}+p_{110}p_{010}+p_{110}p_{100}+p_{110}p_{110}
        -p_{100}p_{010}-p_{100}p_{110}-p_{110}p_{010}-p_{110}p_{110}=
    $$
    $$
        =p_{000}p_{110}-p_{010}p_{100}
    $$
    Таким образом:
    $$
        M_{21} = p_{**0}(p_{001}p_{111}-p_{011}p_{101}) + p_{**1} (p_{000}p_{110}-p_{010}p_{100})
    $$
\end{proof}
\begin{theorem}\label{1.2}
    Пусть $X$ и $Y$ условно независимы при условии $Z$. Тогда $\sigma^{12}=0$.
\end{theorem}
\begin{proof}
    Пусть $X$ и $Y$ условно независимы при условии $Z$. Тогда по теореме \ref{thm1}:
    $$p_{000}p_{110}=p_{010}p_{100}$$
    $$p_{001}p_{111}=p_{011}p_{101}$$
    Используя вышеприведенные соотношения, имеем:
    $$
        M_{21} = p_{**0}(p_{001}p_{111}-p_{011}p_{101}) + p_{**1} (p_{000}p_{110}-p_{010}p_{100})= 0
    $$
    Из $M_{21}=0$ непосредственно следует, что $\sigma^{12}=0$.
\end{proof}
В обратную сторону теорема \ref{1.2} неверна. Легко построить контрпример при $p_{**0}=0$. Далее покажем контрпример в невырожденном случае.
\begin{example}
    Пусть $p_{000}=0.15$, $p_{001}=0.1$, $p_{010}=0.1$, $p_{011}=0.15$, $p_{100}=0.1$, $p_{101}=0.15$, $p_{110}=0.15$, $p_{111}=0.1$.
    Тогда $p_{**0}=0.5$, $p_{**1}=0.5$ и
    $M_{21} = p_{**1}(p_{000}p_{110}-p_{010}p_{100}) + p_{**0}(p_{001}p_{111}-p_{011}p_{101})=$
    $= 0.5 \cdot (0.15 \cdot 0.15 - 0.1 \cdot 0.1) + 0.5 \cdot (0.1 \cdot 0.1 - 0.15 \cdot 0.15) = 0$.

    Однако, случайные величины $X$ и $Y$ условно зависимы при условии $Z$ поскольку:
    $$
        p_{000}p_{110}-p_{010}p_{100}=0.15 \cdot 0.15 - 0.1 \cdot 0.1 = 0.0125 \neq 0
    $$
    $$
        p_{001}p_{111}-p_{011}p_{101}=0.1 \cdot 0.1 - 0.15 \cdot 0.15 = -0.0125 \neq 0
    $$
\end{example}
