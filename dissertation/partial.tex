Пусть $(X,Y,Z)^T$ -- случайный вектор с трехмерным распределение Бернулли,
$\Sigma$ -- ковариационная матрица:
$$\Sigma =
    \begin{pmatrix}
        \sigma_{XX} & \sigma_{XY} & \sigma_{XZ} \\
        \sigma_{YX} & \sigma_{YY} & \sigma_{YZ} \\
        \sigma_{ZX} & \sigma_{ZY} & \sigma_{ZZ}
    \end{pmatrix}
$$
Остатками от $X$ и $Y$ при регрессии на $Z$ будем называть случайные величины:
$$
X^{\prime}=(X-EX)-\dfrac{\sigma_{XZ}}{\sigma_{ZZ}}(Z-EZ)
$$
$$
Y^{\prime}=(Y-EY)-\dfrac{\sigma_{YZ}}{\sigma_{ZZ}}(Z-EZ)
$$
Согласно работе Крамера, частным коэффициентом корреляции Пирсона называется:
$$
\rho^{XY \cdot Z}=\dfrac{E(X^{\prime} Y^{\prime})}{\sqrt{E(X^{\prime})^2 E(Y^{\prime})^2}}
$$
Докажем следующую лемму:
\begin{lemma}
    $$E(X^\prime Y^\prime)=\sigma_{XY}-\dfrac{\sigma_{XZ}\sigma_{YZ}}{\sigma_{ZZ}}$$
\end{lemma}
\begin{proof}
    $$E(X^\prime Y^\prime)=E[((X-EX)-\dfrac{\sigma_{XZ}}{\sigma_{ZZ}}(Z-EZ))((Y-EY)-\dfrac{\sigma_{YZ}}{\sigma_{ZZ}}(Z-EZ))]=$$
    $$=\sigma_{XY}-\dfrac{\sigma_{YZ}}{\sigma_{ZZ}}\sigma_{XZ}-\dfrac{\sigma_{XZ}}{\sigma_{ZZ}}\sigma_{YZ}+
    \dfrac{\sigma_{XZ}\sigma_{YZ}}{\sigma_{ZZ}\sigma_{ZZ}}\sigma_{ZZ}=\sigma_{XY}-\dfrac{\sigma_{XZ} \sigma_{YZ}}{ \sigma_{ZZ}}$$
\end{proof}

Приведем выражение для частного коэффицинта корреляции Пирсона:
\begin{lemma}
    $$
    \rho^{XY \cdot Z}=\dfrac{\sigma_{XY} \sigma_{ZZ} - \sigma_{XZ} \sigma_{YZ}}{\sqrt{\sigma_{XX}\sigma_{ZZ}-
    \sigma_{XZ}^2}\sqrt{\sigma_{YY}\sigma_{ZZ}-\sigma_{YZ}^2}}
    $$
\end{lemma}
\begin{proof}
    $$E(X^\prime Y^\prime)=\sigma_{XY}-\dfrac{\sigma_{XZ}\sigma_{YZ}}{\sigma_{ZZ}}=
    \dfrac{\sigma_{XY}\sigma_{ZZ}-\sigma_{XZ}\sigma_{YZ}}{\sigma_{ZZ}}$$
    $$
    \rho^{XY \cdot Z}=\dfrac{\dfrac{\sigma_{XY}\sigma_{ZZ}-\sigma_{XZ}\sigma_{YZ}}{\sigma_{ZZ}}}
    {\sqrt{
        \dfrac{\sigma_{XX}\sigma_{ZZ}-\sigma_{XZ}\sigma_{XZ}}{\sigma_{ZZ}}
    }\sqrt{
        \dfrac{\sigma_{YY}\sigma_{ZZ}-\sigma_{YZ}\sigma_{YZ}}{\sigma_{ZZ}}
    }}=
    $$
    $$
    =\dfrac{\sigma_{XY} \sigma_{ZZ} - \sigma_{XZ} \sigma_{YZ}}{\sqrt{\sigma_{XX}\sigma_{ZZ}-
    \sigma_{XZ}^2}\sqrt{\sigma_{YY}\sigma_{ZZ}-\sigma_{YZ}^2}}
    $$
\end{proof}

Для удобства в дальнейших выкладках введем следующие обозначения: $$p_{x**}=P(X=x), \; p_{*y*}=P(Y=y), \; p_{**z}=P(Z=z)$$
$$p_{xy*}=P(X=x, Y=y), \; p_{x*z}=P(X=x, Z=z), \; p_{*yz}=P(Y=y, Z=z)$$
Легко проверить, что $\sigma_{XX}= p_{1**}(1-p_{1**})$.

\begin{lemma}
    $$\sigma_{XY}=p_{11*}-p_{1**}p_{*1*}$$
\end{lemma}

\begin{proof}
    Воспользуемся формулой $\sigma_{XY}=\text{Cov}(X,Y)=E(X Y)-$ $-E(X)E(Y)$.
    $$E(X Y) = 1 \cdot p_{11*} + 0 \cdot (p_{00*} + p_{01*} + p_{10*})=p_{11*}$$
    $$EX = 1 \cdot p_{1**} + 0 \cdot p_{0**}=p_{1**}$$
    $$ EY = 1 \cdot p_{*1*} + 0 \cdot p_{*0*} = p_{*1*}$$
    Таким образом, $\text{Cov}(X,Y)=p_{11*}-p_{1**}p_{*1*}$.
\end{proof}

Докажем лемму касательно выражения, фигурирующего в числителе частного коэффициента корреляции Пирсона.

\begin{lemma}\label{partial_cov}
    $$\sigma_{XY} \sigma_{ZZ} - \sigma_{XZ} \sigma_{YZ} = p_{**0}(p_{001}p_{111}-p_{011}p_{101}) + p_{**1} (p_{000}p_{110}-p_{010}p_{100})$$
\end{lemma}
\begin{proof}
    $$ \sigma_{XY} \sigma_{ZZ} - \sigma_{XZ} \sigma_{YZ}
        = (p_{11*}-p_{1**}p_{*1*}) p_{**1}(1-p_{**1})-
    $$
    $$
        -(p_{1*1}-p_{1**}p_{**1})(p_{*11}-p_{*1*}p_{**1})=
    $$
    $$
        =p_{11*}p_{**1} - p_{11*}p_{**1}p_{**1} - p_{1**}p_{*1*}p_{**1} + p_{1**}p_{*1*}p_{**1}p_{**1}-
    $$
    $$
        -p_{1*1}p_{*11}+p_{1*1}p_{*1*}p_{**1}+p_{1**}p_{**1}p_{*11}-p_{1**}p_{**1}p_{*1*}p_{**1}=
    $$
    Заметим, что четвертое и восьмое слагаемые сокращаются. Распишем первое слагаемое как сумму вероятностей по компоненте $z$:
    $$
        =p_{111}p_{**1}+p_{110}p_{**1} - p_{11*}p_{**1}p_{**1} - p_{1**}p_{*1*}p_{**1} -
    $$
    $$
        -p_{1*1}p_{*11}+p_{1*1}p_{*1*}p_{**1}+p_{1**}p_{**1}p_{*11}=
    $$
    Осуществим перегруппировку слагаемых:
    $$
        =(p_{111}p_{**1}-p_{1*1}p_{*11})+p_{**1}(p_{110}-p_{11*}p_{**1} - p_{1**}p_{*1*} + p_{1*1}p_{*1*} + p_{1**}p_{*11})
    $$
    Преобразуем выражения для отдельных слагаемых.
    Заметим, что:
    $$
        p_{110}-p_{11*}p_{**1}=p_{110}-p_{110}p_{**1}-p_{111}p_{**1}=
    $$
    $$
        =p_{110}(1-p_{**1})-p_{111}p_{**1}=p_{110}p_{**0}-p_{111}p_{**1}
    $$
    Также заметим, что:
    $$
        -p_{1**}p_{*1*} + p_{1*1}p_{*1*} + p_{1**}p_{*11}=
    $$
    $$
        =-(p_{1*0}+p_{1*1})(p_{*10}+p_{*11})+p_{1*1}(p_{*10}+p_{*11}) + (p_{1*0}+p_{1*1})p_{*11}=
    $$
    $$
        =-p_{1*0}p_{*10}-p_{1*0}p_{*11}-p_{1*1}p_{*10}-p_{1*1}p_{*11}+
    $$
    $$
        +p_{1*1}p_{*10}+p_{1*1}p_{*11}+p_{1*0}p_{*11}+p_{1*1}p_{*11}=
    $$
    $$
        =-p_{1*0}p_{*10}+p_{1*1}p_{*11}
    $$
    Запишем выражение для $\sigma_{XY} \sigma_{ZZ} - \sigma_{XZ} \sigma_{YZ}$ с преобразованными слагаемыми:
    $$
    \sigma_{XY} \sigma_{ZZ} - \sigma_{XZ} \sigma_{YZ}=
    $$
    $$
    =(p_{111}p_{**1}-p_{1*1}p_{*11})+p_{**1}((p_{110}p_{**0}-p_{1*0}p_{*10})-(p_{111}p_{**1}-p_{1*1}p_{*11}))=
    $$
    $$
        =(p_{111}p_{**1}-p_{1*1}p_{*11})+p_{**1}(p_{110}p_{**0}-p_{1*0}p_{*10})-p_{**1}(p_{111}p_{**1}-p_{1*1}p_{*11})=
    $$
    $$
        =(1-p_{**1})(p_{111}p_{**1}-p_{1*1}p_{*11})+p_{**1}(p_{110}p_{**0}-p_{1*0}p_{*10})=
    $$
    $$
        =p_{**0}(p_{111}p_{**1}-p_{1*1}p_{*11})+p_{**1}(p_{110}p_{**0}-p_{1*0}p_{*10})
    $$
    Снова преобразуем отдельные слагаемые.
    $$
        p_{111}p_{**1}-p_{1*1}p_{*11} = p_{111}(p_{001}+p_{011}+p_{101}+p_{111})-
        (p_{101}+p_{111})(p_{011}+p_{111})=
    $$
    $$
        = p_{111}p_{001}+p_{111}p_{011}+p_{111}p_{101}+p_{111}p_{111}
        - p_{101}p_{011}-p_{101}p_{111}-p_{111}p_{011}-p_{111}p_{111}=
    $$
    $$
        = p_{001}p_{111}-p_{011}p_{101}
    $$
    Аналогично преобразуем выражение:
    $$
        p_{110}p_{**0}-p_{1*0}p_{*10}=
        p_{110}(p_{000}+p_{010}+p_{100}+p_{110})-(p_{100}+p_{110})(p_{010}+p_{110})=
    $$
    $$
        =p_{110}p_{000}+p_{110}p_{010}+p_{110}p_{100}+p_{110}p_{110}
        -p_{100}p_{010}-p_{100}p_{110}-p_{110}p_{010}-p_{110}p_{110}=
    $$
    $$
        =p_{000}p_{110}-p_{010}p_{100}
    $$
    Таким образом:
    $$
    \sigma_{XY} \sigma_{ZZ} - \sigma_{XZ} \sigma_{YZ} = p_{**0}(p_{001}p_{111}-p_{011}p_{101}) + p_{**1} (p_{000}p_{110}-p_{010}p_{100})
    $$
\end{proof}
\begin{theorem}\label{1.2}
    Пусть $X$ и $Y$ условно независимы при условии $Z$. Тогда $\rho^{XY \cdot Z}=0$.
\end{theorem}
\begin{proof}
    Пусть $X$ и $Y$ условно независимы при условии $Z$. Тогда по теореме \ref{thm1}:
    $$p_{000}p_{110}=p_{010}p_{100}$$
    $$p_{001}p_{111}=p_{011}p_{101}$$
    Используя вышеприведенные соотношения в числителе частного коэффициента корреляции Пирсона имеем:
    $$
        p_{**0}(p_{001}p_{111}-p_{011}p_{101}) + p_{**1} (p_{000}p_{110}-p_{010}p_{100})= 0
    $$
    Следовательно, $\rho^{XY \cdot Z}=0$.
\end{proof}
В обратную сторону теорема \ref{1.2} неверна. Легко построить контрпример при $p_{**0}=0$. Далее покажем контрпример в невырожденном случае.
\begin{example}
    Пусть $p_{000}=0.15$, $p_{001}=0.1$, $p_{010}=0.1$, $p_{011}=0.15$, $p_{100}=0.1$, $p_{101}=0.15$, $p_{110}=0.15$, $p_{111}=0.1$.
    Тогда $p_{**0}=0.5$, $p_{**1}=0.5$ и
    $M_{21} = p_{**1}(p_{000}p_{110}-p_{010}p_{100}) + p_{**0}(p_{001}p_{111}-p_{011}p_{101})=$
    $= 0.5 \cdot (0.15 \cdot 0.15 - 0.1 \cdot 0.1) + 0.5 \cdot (0.1 \cdot 0.1 - 0.15 \cdot 0.15) = 0$.

    Однако, случайные величины $X$ и $Y$ условно зависимы при условии $Z$ поскольку:
    $$
        p_{000}p_{110}-p_{010}p_{100}=0.15 \cdot 0.15 - 0.1 \cdot 0.1 = 0.0125 \neq 0
    $$
    $$
        p_{001}p_{111}-p_{011}p_{101}=0.1 \cdot 0.1 - 0.15 \cdot 0.15 = -0.0125 \neq 0
    $$
\end{example}

Приведем альтернативные формулы, с помощью которых удобно вычислять частный коэффициент корреляции Пирсона.

\begin{definition}
    Коэффициентом корреляции Пирсона называется:
    $$
    \rho_{XY}=\dfrac{\sigma_{XY}}{\sqrt{\sigma_{XX}\sigma_{YY}}}
    $$
\end{definition}

\begin{lemma}
    $$
    \rho^{XY \cdot Z}=\dfrac{\rho_{XY}-\rho_{XZ}\rho_{YZ}}{\sqrt{1-\rho_{XZ}^2}\sqrt{1-\rho_{YZ}^2}}
    $$
\end{lemma}
\begin{proof}
    $$
    \rho^{XY \cdot Z}=\dfrac{\sigma_{XY} \sigma_{ZZ} - \sigma_{XZ} \sigma_{YZ}}{\sqrt{\sigma_{XX}\sigma_{ZZ}-
    \sigma_{XZ}^2}\sqrt{\sigma_{YY}\sigma_{ZZ}-\sigma_{YZ}^2}}
    $$
    $$
    =\dfrac{\sigma_{ZZ}\sqrt{\sigma_{XX}\sigma_{YY}}
    \left(\dfrac{\sigma_{XY}}{\sqrt{\sigma_{XX}\sigma_{YY}}} 
    - \dfrac{\sigma_{XZ}}{\sqrt{\sigma_{XX}\sigma_{ZZ}}} \dfrac{\sigma_{YZ}}{\sqrt{\sigma_{YY}\sigma_{ZZ}}}\right)}
    {\sqrt{\sigma_{XX}\sigma_{ZZ}-
    \sigma_{XZ}^2}\sqrt{\sigma_{YY}\sigma_{ZZ}-\sigma_{YZ}^2}}=
    $$
    $$
    =\dfrac{\rho_{XY}-\rho_{XZ}\rho_{YZ}}{\sqrt{1-\dfrac{\sigma_{XZ}^2}{\sigma_{XX}\sigma_{ZZ}}}
    \sqrt{1-\dfrac{\sigma_{YZ}^2}{\sigma_{YY}\sigma_{ZZ}}}}
    =\dfrac{\rho_{XY}-\rho_{XZ}\rho_{YZ}}{\sqrt{1-\rho_{XZ}^2}\sqrt{1-\rho_{YZ}^2}}
    $$
\end{proof}

\begin{lemma}
    Пусть $\Sigma$ -- ковариационная матрица с элементами
    $$\Sigma =
    \begin{pmatrix}
        \sigma_{XX} & \sigma_{XY} & \sigma_{XZ} \\
        \sigma_{YX} & \sigma_{YY} & \sigma_{YZ} \\
        \sigma_{ZX} & \sigma_{ZY} & \sigma_{ZZ}
    \end{pmatrix}
    $$
    а $\Sigma^{-1}$ обратная ковариационная матрица с элементами
    $$\Sigma^{-1} =
    \begin{pmatrix}
        \sigma^{XX} & \sigma^{XY} & \sigma^{XZ} \\
        \sigma^{YX} & \sigma^{YY} & \sigma^{YZ} \\
        \sigma^{ZX} & \sigma^{ZY} & \sigma^{ZZ}
    \end{pmatrix}
    $$
    Тогда для частного коэффициента корреляции справедливо:
    $$
    \rho^{XY \cdot Z}=-\dfrac{\sigma^{XY}}{\sqrt{\sigma^{XX}\sigma^{YY}}}
    $$
\end{lemma}
\begin{proof}
    Воспользуемся следующими соотношениями для элементов обратной матрицы:
    $$
    \sigma^{XY}=\dfrac{-1}{det(\Sigma)}\begin{vmatrix}
    \sigma_{XY} & \sigma_{XZ}\\
    \sigma_{YZ} & \sigma_{ZZ}
    \end{vmatrix}=\dfrac{-(\sigma_{XY}\sigma_{ZZ}-\sigma_{XZ}\sigma_{YZ})}{det(\Sigma)}
    $$

    $$
    \sigma^{XX}=\dfrac{1}{det(\Sigma)}\begin{vmatrix}
    \sigma_{YY} & \sigma_{YZ}\\
    \sigma_{YZ} & \sigma_{ZZ}
    \end{vmatrix}
    = \dfrac{\sigma_{YY}\sigma_{ZZ}-\sigma_{YZ}^2}{det(\Sigma)}
    $$

    $$
    \sigma^{YY}=\dfrac{1}{det(\Sigma)}\begin{vmatrix}
    \sigma_{XX} & \sigma_{XZ}\\
    \sigma_{XZ} & \sigma_{ZZ}
    \end{vmatrix}
    = \dfrac{\sigma_{XX}\sigma_{ZZ}-\sigma_{XZ}^2}{det(\Sigma)}
    $$
    Тогда:
    $$
    -\dfrac{\sigma^{XY}}{\sqrt{\sigma^{XX}\sigma^{YY}}}=-\dfrac{
        \dfrac{-(\sigma_{XY}\sigma_{ZZ}-\sigma_{XZ}\sigma_{YZ})}{det(\Sigma)}
    }{
        \sqrt{\dfrac{\sigma_{XX}\sigma_{ZZ}-\sigma_{XZ}^2}{det(\Sigma)}}
        \sqrt{\dfrac{\sigma_{YY}\sigma_{ZZ}-\sigma_{YZ}^2}{det(\Sigma)}}
    }=
    $$
    $$
    = \dfrac{\sigma_{XY}\sigma_{ZZ}-\sigma_{XZ}\sigma_{YZ}}{
        \sqrt{\sigma_{XX}\sigma_{ZZ}-\sigma_{XZ}^2}
        \sqrt{\sigma_{YY}\sigma_{ZZ}-\sigma_{YZ}^2}
    }=\rho^{XY \cdot Z}
    $$
\end{proof}