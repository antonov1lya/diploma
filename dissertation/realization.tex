\subsection{Способ вычисления вероятностей для РНМН теста
на ЭВМ}

Для нахождения порогов в РНМН тестах возникает необходимость
посчета вероятностей вида:
$$P_{\theta_0}(U=u \mid T=t)=\dfrac{(\prod_{i=1}^p k_i(u)!)^{-1}}
            {\sum_{s\in \mathcal{D}} (\prod_{i=1}^p k_i(s)!)^{-1}}$$
где $\mathcal{D}$ -- некая область допустимых значений,
$k_i(u):\mathcal{D} \to \{0,\ldots,n\}, \; i=\overline{1,p}$.
Основную проблему в этой формуле представляют 
факториалы, вычисление которых затруднительно на ЭВМ. 
Предложим методологию, которая поможет обойти эту проблему.

    Пусть $f(i)=\sum_{j=1}^{i} \ln(j)$. Тогда $\ln(n!)=f(n)$.
    Учитывая это, запишем:
    $$
    \dfrac{(\prod_{i=1}^p k_i(u)!)^{-1}}
            {\sum_{s\in \mathcal{D}} (\prod_{i=1}^p k_i(s)!)^{-1}}=
    \dfrac{\exp\{-\ln(\prod_{i=1}^p k_i(u)!)\}}
    {\sum_{s\in \mathcal{D}} \exp \{-\ln(\prod_{i=1}^p k_i(s)!)\}}=
    $$
    $$
    = \dfrac{\exp \{ -\sum_{i=1}^p f(k_i(u)) \}}
    {\sum_{s \in D} \exp \{ -\sum_{i=1}^p f(k_i(s)) \}}
    $$
    
    Полученное выражение удобно с позиции того, что
    мы ушли от вычисления факториалов и ЭВМ умеют 
    эффективно считать функцию:
    $$
    \text{softmax}(x,i)=\dfrac{\exp\{x_i\}}{\sum_{j=1}^{N} \exp\{x_j\}}, \; x=(x_1,\ldots,x_N)
    $$
    Это происходит благодаря свойству:
    $$
    \text{softmax}(x,i)=\dfrac{\exp\{x_i\}}{\sum_{j=1}^{N} \exp\{x_j\}} = \dfrac{\exp\{x_i - C\}}{\sum_{j=1}^{N} \exp\{x_j - C\}}
    , \text{ где } C=\max_{1\leq j \leq N} x_j
    $$
    за счет которого удается избежать переполнения вещественного типа данных, 
    связанного с вычислением экспоненты.
