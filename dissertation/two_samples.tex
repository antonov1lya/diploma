\subsection{Процедура проверки условной независимости
в трехмерном распределении Бернулли}

Пусть $(X,Y,Z)^T$ -- случайный вектор, имеющий трехмерное распределение
Бернулли, а
$$
\begin{pmatrix}
        x_1 \\
        y_1 \\
        z_1
    \end{pmatrix},
    \begin{pmatrix}
        x_2 \\
        y_2 \\
        z_2
    \end{pmatrix}, \ldots,
    \begin{pmatrix}
        x_n \\
        y_n \\
        z_n
    \end{pmatrix}
$$ реализация повторной выборки из распределения $(X,Y,Z)^T$.

Предложим процедуру проверки условной независимости.
\begin{itemize}
    \item Разобьем исходную выборку на две подвыборки:
    $$
    \begin{pmatrix}
            x_{i_1} \\
            y_{i_1} \\
            0
        \end{pmatrix},
        \begin{pmatrix}
            x_{i_2} \\
            y_{i_2} \\
            0
        \end{pmatrix}, \ldots,
        \begin{pmatrix}
            x_{i_{n_0}} \\
            y_{i_{n_0}} \\
            0
        \end{pmatrix}
    \text{ и }
    \begin{pmatrix}
            x_{j_1} \\
            y_{j_1} \\
            1
        \end{pmatrix},
        \begin{pmatrix}
            x_{j_2} \\
            y_{j_2} \\
            1
        \end{pmatrix}, \ldots,
        \begin{pmatrix}
            x_{j_{n_1}} \\
            y_{j_{n_1}} \\
            1
        \end{pmatrix}
    $$
    \item По наблюдениям 
    $$
    \begin{pmatrix}
        x_{i_1} \\
        y_{i_1} 
    \end{pmatrix},
    \begin{pmatrix}
        x_{i_2} \\
        y_{i_2}
    \end{pmatrix}, \ldots,
    \begin{pmatrix}
        x_{i_{n_0}} \\
        y_{i_{n_0}}
    \end{pmatrix}
    $$
тестом $\varphi_0=\varphi^{\text{Independence}}_0$ уровня $\gamma$
проверим гипотезу $H_0 : X$ и $Y$ независимы при условии $Z=0$.
Если подвыборка не содержит наблюдений, то применяем тест 
$\varphi \equiv \gamma$.
\item По наблюдениям 
    $$
    \begin{pmatrix}
        x_{j_1} \\
        y_{j_1} 
    \end{pmatrix},
    \begin{pmatrix}
        x_{j_2} \\
        y_{j_2}
    \end{pmatrix}, \ldots,
    \begin{pmatrix}
        x_{j_{n_1}} \\
        y_{j_{n_1}}
    \end{pmatrix}
    $$
тестом $\varphi_1=\varphi^{\text{Independence}}_1$ уровня $\gamma$
проверим гипотезу $H_1 : X$ и $Y$ независимы при условии $Z=1$.
Если подвыборка не содержит наблюдений, то применяем тест 
$\varphi \equiv \gamma$.
\item 
Для проверки гипотезы $H: X \ci Y \mid Z$ используем тест:
$$
\varphi^{\text{Subsamples}}=\begin{cases}
    1, \text{ наступило событие $A_0 \cup A_1$}\\
    0, \text{ иначе}
\end{cases}
$$
где $A_0 = \{\text{гипотеза $H_0$ отвергнута}\},\;
A_1 = \{\text{гипотеза $H_1$ отвергнута}\}$.
\end{itemize}
Очевидно, что при истинности гипотезы $H: X \ci Y \mid Z$,
события $A_0$ и $A_1$ независимые. Поэтому для контроля
$P_{H}(\varphi^{\text{Subsamples}}=1)=\alpha$ достаточно положить
$\gamma = 1 - \sqrt{1-\alpha}$.

% \newpage
% \subsection{Проверка условной независимости по подвыборкам из условных распределений}\label{twos}

% Приведем трактовку \autoref{cond_ind_def} для трехмерного
% распределения Бернулли. 
% \begin{definition}\label{new_def}
%     В трехмерном распределении Бернулли случайные величины $X$ и $Y$
%     условно независимы при условии $Z$, 
%     если выполнены следующие условия:
%     \begin{itemize}
%         \item $X$ и $Y$ независимы при условии $Z=0$
%         \item $X$ и $Y$  независимы при условии $Z=1$
%     \end{itemize}
% \end{definition}
% Используя \autoref{new_def}, сформулируем индивидуальные гипотезы:
% \begin{itemize}
%     \item $H_0 : X$ и $Y$ независимы при условии $Z=0$
%     \item $H_1 : X$ и $Y$ независимы при условии $Z=1$
% \end{itemize}
% Тогда гипотеза об условной независимости имеет вид 
% $H = H_0 \cap H_1$. Естественным образом, гипотезу 
% $H_0$ необходимо проверять по наблюдениям
% $(x_i,y_i,z_i)^T$, в которых $z_i=0$. Поскольку $(X,Y)^T$ при условии
% $Z=z$ имеет двумерное распределение Бернулли, то
% в качестве теста для $H_0$ можно использовать
% $\varphi_0 = \varphi^{\text{Independence}}_0$, 
% приведенный в \autoref{bivariate_umpu}. Аналогичные рассуждения 
% справедливы и для гипотезы $H_1$. Учитывая озвученные соображения,
% построим тест проверки гипотезы $H$, контролирующий вероятность
% ошибки первого рода на уровне $\alpha$.
% Пусть
% $$
% \begin{pmatrix}
%         X_1 \\
%         Y_1 \\
%         Z_1
%     \end{pmatrix},
%     \begin{pmatrix}
%         X_2 \\
%         Y_2 \\
%         Z_2
%     \end{pmatrix}, \ldots,
%     \begin{pmatrix}
%         X_n \\
%         Y_n \\
%         Z_n
%     \end{pmatrix}
% $$ повторная выборка из распределения случайного вектора $(X,Y,Z)^T$. Обозначим $\mathbf{Z}=(Z_1,\ldots,Z_n)$ и 
% $\mathbf{z}=(z_1,\ldots,z_n)$.
% Покажем, что в условном распределении при $\mathbf{Z}=\mathbf{z}$ наблюдения
% $$
% \Xi = 
% \begin{pmatrix}
%         X_1 \\
%         Y_1
%     \end{pmatrix},
%     \begin{pmatrix}
%         X_2 \\
%         Y_2
%     \end{pmatrix}, \ldots,
%     \begin{pmatrix}
%         X_n \\
%         Y_n
%     \end{pmatrix}
% $$
% являются независимыми. 

% \begin{lemma}\label{ci_for_samples}
%     $$
%     P(X_1=x_1,Y_1=y_1,\ldots,X_n=x_n,Y_n=y_n \mid \mathbf{Z}=\mathbf{z})=
%     $$
%     $$
%     =\prod_{i=1}^n P(X_i=x_i, Y_i=y_i \mid \mathbf{Z}=\mathbf{z})
%     $$
% \end{lemma}
% \begin{proof}
%     С одной стороны:
%     $$
%     P(X_1=x_1,Y_1=y_1,Z_1=z_1,\ldots,X_n=x_n,Y_n=y_n,Z_n=z_n)=
%     $$
%     $$
%     =P(\mathbf{Z}=\mathbf{z}) P(X_1=x_1,Y_1=y_1,\ldots,X_n=x_n,Y_n=y_n \mid \mathbf{Z}=\mathbf{z}) 
%     $$
%     С другой стороны:
%     $$
%     P(X_1=x_1,Y_1=y_1,Z_1=z_1,\ldots,X_n=x_n,Y_n=y_n,Z_n=z_n)=
%     $$
%     $$
%     =\prod_{i=1}^n P(X_i=x_i, Y_i=y_i, Z_i=z_i)=
%     $$
%     $$
%     =\prod_{i=1}^n P(X_i=x_i, Y_i=y_i \mid Z_i=z_i)P(Z_i=z_i)=
%     $$
%     $$
%     =P(\mathbf{Z}=\mathbf{z})\prod_{i=1}^n P(X_i=x_i, Y_i=y_i \mid \mathbf{Z}=\mathbf{z})
%     $$
% \end{proof}

% Пусть также $\mathbf{Z}=\mathbf{z}$ фиксированы. 
% Разобьем выборку $\Xi$ на две подвыборки $\Xi_0$ и $\Xi_1$, 
% такие что:
% $$
% \Xi_0=
% \begin{pmatrix}
%     X_{i_1} \\
%     Y_{i_1} \\
% \end{pmatrix},
% \begin{pmatrix}
%     X_{i_2} \\
%     Y_{i_2} \\
% \end{pmatrix}, \ldots,
% \begin{pmatrix}
%     X_{i_{n_0}} \\
%     Y_{i_{n_0}} \\
% \end{pmatrix} 
% $$
% где $Z_{i_k}=z_{i_k}=0$ для всех $i_k$ при $k=\overline{1,n_0}$ и $$
% \Xi_1=
% \begin{pmatrix}
%     X_{j_1} \\
%     Y_{j_1} \\
% \end{pmatrix},
% \begin{pmatrix}
%     X_{j_2} \\
%     Y_{j_2} \\
% \end{pmatrix}, \ldots,
% \begin{pmatrix}
%     X_{j_{n_1}} \\
%     Y_{j_{n_1}} \\
% \end{pmatrix} 
% $$
% где $Z_{j_k}=z_{j_k}=1$ для всех $j_k$ при $k=\overline{1,n_1}$. 
% Причем $n=n_0+n_1$. Отметим, что
% разбиение $\Xi = \Xi_0 \sqcup \Xi_1$ определяется лишь набором
% $\mathbf{Z}=\mathbf{z}$.

% Сформулируем следующую теорему.

% \begin{theorem}
%     Пусть $\mathbf{Z}=\mathbf{z}$ -- фиксированы. Тогда выборка $\Xi_0$ является повторной выборкой из распределения $(X,Y)^T$ при условии $Z=0$.
% \end{theorem}
% \begin{proof}
%     По \autoref{ci_for_samples}:
%     $$
%     P(X_1=x_1,Y_1=y_1,\ldots,X_n=x_n,Y_n=y_n \mid \mathbf{Z}=\mathbf{z})=
%     $$
%     $$
%     =\prod_{i=1}^n P(X_i=x_i, Y_i=y_i \mid \mathbf{Z}=\mathbf{z})
%     $$
%     Просуммировав обе части вышепредставленного равенства по всем возможным значениям 
%     $x_{j_1}, y_{j_1}, \ldots, x_{j_{n_1}},y_{j_{n_1}}$ получаем:
%     $$
%     P(X_{i_1}=x_{i_1},Y_{i_1}=y_{i_1},\ldots,X_{i_{n_0}}=x_{i_{n_0}},Y_{i_{n_0}}=y_{i_{n_0}} \mid \mathbf{Z}=\mathbf{z})=
%     $$
%     $$
%     =\prod_{k=1}^{n_0} P(X_{i_k}=x_{i_k}, Y_{i_k}=y_{i_k} \mid \mathbf{Z}=\mathbf{z})
%     $$
%     Значит $$
%     \Xi_0=
%     \begin{pmatrix}
%         X_{i_1} \\
%         Y_{i_1} \\
%     \end{pmatrix},
%     \begin{pmatrix}
%         X_{i_2} \\
%         Y_{i_2} \\
%     \end{pmatrix}, \ldots,
%     \begin{pmatrix}
%         X_{i_{n_0}} \\
%         Y_{i_{n_0}} \\
%     \end{pmatrix} 
%     $$
%     независимые наблюдения при условии $\mathbf{Z}=\mathbf{z}$.

%     Покажем, что $(X_{i_k},Y_{i_k})^T$ при условии $\mathbf{Z}=\mathbf{z}$ распределен также как и 
%     $(X,Y)^T$ при условии $Z=0$.
%     $$
%     P(X_{i_k}=x_{i_k}, Y_{i_k}=y_{i_k} \mid \mathbf{Z}=\mathbf{z}) =
%     P(X_{i_k}=x_{i_k}, Y_{i_k}=y_{i_k} \mid Z_{i_k}=z_{i_k}) = $$
%     $$=P(X_{i_k}=x_{i_k}, Y_{i_k}=y_{i_k} \mid Z_{i_{k}}=0)=
%     P(X=x_{i_k}, Y=y_{i_k} \mid Z=0)$$
% \end{proof}

% Аналогично показывается, что $\Xi_1$ является повторной выборкой из распределения $(X,Y)^T$ при условии $Z=1$.

% Сформулируем теорему.
% \begin{theorem}\label{main_theorem}
%     Пусть $\Xi_0$ и $\Xi_1$ -- подвыборки, полученные разбиением случайной выборки $\Xi$.
%     Пусть $\varphi_0$ и $\varphi_1$ -- рандомизированные тесты 
%     проверки гипотез $H_0$ и $H_1$ по повторным выборкам 
%     $\Xi_0$ и $\Xi_1$ соответственно.
%     Введем события:
%     $$A_0 = \{\text{отвергнуть гипотезу $H_0$ рандомизированным тестом $\varphi_0$}\}$$ 
%     $$A_1 = \{\text{отвергнуть гипотезу $H_1$ рандомизированным тестом $\varphi_1$}\}$$
%     Пусть $\varphi_0$ и $\varphi_1$ тесты уровня $\alpha_0$ и $\alpha_1$
%     при любом объеме наблюдений в подвыборках $\Xi_0$ и $\Xi_1$ соответственно, то есть:
%     $$P_{H_0\cap H_1,\mathbf{Z=z}}(A_0)=\alpha_0$$ 
%     $$P_{H_0\cap H_1,\mathbf{Z=z}}(A_1)=\alpha_1$$
%     Тогда $P_{H_0\cap H_1}(A_0 \cap A_1)= P_{H_0\cap H_1}(A_0) P_{H_0\cap H_1}(A_1)$.
% \end{theorem}
% \begin{proof}
%     Пусть $\mathbf{Z}=\mathbf{z}$ фиксировано. 
%     Обозначим через
%     $$T_0=(X_{i_1},Y_{i_1},\ldots,X_{i_{n_0}},Y_{i_{n_0}})^T, \;
%     T_1=(X_{j_1},Y_{j_1},\ldots,X_{j_{n_1}},Y_{j_{n_1}})^T$$
%     случайные векторы наблюдений, используемые в тестах $\varphi_0$ и 
%     $\varphi_1$ соответственно.
%     Распишем следующую вероятность
%     $$
%     P_{H_0\cap H_1,\mathbf{Z=z}}(A_0 \cap A_1)=
%     $$
%     $$
%     =\sum_{t_0}\sum_{t_1} P_{H_0\cap H_1,\mathbf{Z=z}}(A_0 \cap A_1 \mid T_0=t_0, T_1=t_1)P_{H_0\cap H_1,\mathbf{Z=z}}(T_0=t_0, T_1=t_1)
%     $$
%     Отметим, что $P_{H_0\cap H_1,\mathbf{Z=z}}(A_0 \cap A_1 \mid T_0=t_0, T_1=t_1)=\varphi_0(t_0)\varphi_1(t_1)$, поскольку
%     для того, чтобы при известных значениях статистик $t_0$ и $t_1$ отвергнуть гипотезы $H_0$ и $H_1$, в рандомизированном тесте нужно провести два испытания с вероятностью успеха
%     $\varphi_0(t_0)$ и $\varphi_1(t_1)$. Постулируется, что такие испытания независимые. Тогда:
%     $$
%     P_{H_0\cap H_1,\mathbf{Z=z}}(A_0 \cap A_1)=\sum_{t_0}\sum_{t_1} \varphi_0(t_0) \varphi_1(t_1) P_{H_0\cap H_1,\mathbf{Z=z}}(T_0=t_0, T_1=t_1)=
%     $$
%     $$
%     =\sum_{t_0}\sum_{t_1} \varphi_0(t_0) \varphi_1(t_1) P_{H_0\cap H_1,\mathbf{Z=z}}(T_0=t_0)P_{H_0\cap H_1,\mathbf{Z=z}}(T_1=t_1)=
%     $$
%     $$
%     =\sum_{t_0}\left[ \varphi_0(t_0) P_{H_0\cap H_1,\mathbf{Z=z}}(T_0=t_0) \left(\sum_{t_1}\varphi_1(t_1) P_{H_0\cap H_1,\mathbf{Z=z}}(T_1=t_1)\right)\right]=
%     $$
%     $$
%     = \sum_{t_0} \varphi_0(t_0) P_{H_0\cap H_1,\mathbf{Z=z}}(T_0=t_0) \alpha_1 
%     =\alpha_0 \alpha_1
%     $$
%     По формуле полной вероятности:
%     $$
%     P_{H_0\cap H_1}(A_0) = \sum_{\mathbf{z}} P_{H_0\cap H_1,\mathbf{Z=z}}(A_0) P_{H_0\cap H_1}(\mathbf{Z}=\mathbf{z})
%     = \sum_{\mathbf{z}} \alpha_0 P_{H_0\cap H_1}(\mathbf{Z}=\mathbf{z})=\alpha_0
%     $$
%     Аналогично, $P_{H_0\cap H_1}(A_1)=\alpha_1$. Также по формуле полной вероятности:
%     $$
%     P_{H_0\cap H_1}(A_0 \cap A_1) = \sum_{\mathbf{z}} P_{H_0\cap H_1,\mathbf{Z=z}}(A_0 \cap A_1) P_{H_0\cap H_1}(\mathbf{Z}=\mathbf{z})=
%     $$
%     $$
%     = \sum_{\mathbf{z}} \alpha_0 \alpha_1 P_{H_0\cap H_1}(\mathbf{Z}=\mathbf{z})
%     = \alpha_0 \alpha_1
%     $$
%     Таким образом, $P_{H_0\cap H_1}(A_0 \cap A_1)=P_{H_0\cap H_1}(A_0) P_{H_0\cap H_1}(A_1)=\alpha_0 \alpha_1$.
% \end{proof}
% Применим \autoref{main_theorem} для проверки условной независимости в трехмерном
% распределении Бернулли.
% Положим индивидуальные гипотезы:
% \begin{itemize}
%     \item $H_0 : X$ и $Y$ независимы при условии $Z=0$
%     \item $H_1 : X$ и $Y$ независимы при условии $Z=1$
% \end{itemize}
% Для проверки гипотез $H_0$ и $H_1$ будем использовать тесты
% $\varphi_0 = \varphi^{\text{Independence}}_0$ и 
% $\varphi_1 = \varphi^{\text{Independence}}_1$ уровня $\alpha_0$ и $\alpha_1$ 
% по повторным выборкам
% $\Xi_0$ и $\Xi_1$ соответственно.
% Тогда гипотеза
% об условной независимости имеет вид $H = H_0 \cap H_1$ и тест проверки условной независимости можно определить как:
% $$
% \varphi^{\text{Subsamples}}=\begin{cases}
%     1, \text{ наступило событие $A_0 \cup A_1$}\\
%     0, \text{ иначе}
% \end{cases}
% $$
% Пусть далее $\alpha_0 = \alpha_1 = \gamma$.
% Тогда $$P_{H_0\cap H_1}(\varphi^{\text{Subsamples}}=1)=P_{H_0\cap H_1}(A_0 \cup A_1)=$$ 
% $$ = P_{H_0\cap H_1}(A_0) + P_{H_0\cap H_1}(A_1) - 
% P_{H_0\cap H_1}(A_0 \cap A_1) = 2\gamma - \gamma^2$$
% Нетрудно проверить, что для контроля $P_{H_0\cap H_1}(\varphi^{\text{Subsamples}}=1)=\alpha$ достаточно положить
% уровень значимости $\gamma = 1 - \sqrt{1-\alpha}$ на
% тестах проверки индивидуальных гипотез $H_0$ и $H_1$.

% Покажем, что тест $\varphi^{\text{Subsamples}}$ является несмещенным.
% \begin{theorem}\label{unbias}
%     Тест $\varphi^{\text{Subsamples}}$
%     уровня $\alpha$
%     проверки гипотезы 
%     $H: X \ci Y \mid Z$ является несмещенным.
% \end{theorem}
% \begin{proof}
%     Положим $\Theta_{H}=\{\theta: p_{000}p_{110}=p_{010}p_{100}
%     \text{ и } p_{001}p_{111}=p_{011}p_{101}\}$, 
%     $\Theta_{K}=\{\theta: p_{000}p_{110}\neq p_{010}p_{100}
%     \text{ или } p_{001}p_{111}\neq p_{011}p_{101}\}$.

%     Отметим, что $\varphi_0 = \varphi^{\text{Independence}}_0$ и
%     $\varphi_1 = \varphi^{\text{Independence}}_1$ -- несмещенные тесты
%     проверки гипотез  $H_0$ и $H_1$ уровня $\gamma$
%     соответственно. Причем, при истинности гипотез $H_0$
%     и $H_1$ на тестах $\varphi_0$ и $\varphi_1$ вероятность ошибки
%     первого рода в точности равна $\gamma$.
%     Поэтому,
%     $E_{\theta}(\varphi_0) = P_\theta(A_0)\geq \gamma$ и $E_{\theta}(\varphi_1) = P_\theta(A_1)\geq \gamma$ для любого
%     $\theta \in \Theta_{H} \cup \Theta_{K}$.

%     Пусть $\theta \in \Theta_{H}$, тогда
%     $E_{\theta}(\varphi^{\text{Subsamples}}) =
%     P_{\theta}(\varphi^{\text{Subsamples}}=1) = \alpha$ поскольку
%     $\varphi^{\text{Subsamples}}$ тест уровня $\alpha$.

%     Пусть $\theta \in \Theta_{K}$.
%     Как и в доказательстве \autoref{main_theorem} можно показать, что
%     $P_{\theta,\mathbf{Z}=\mathbf{z}}(A_0 \cap A_1)=
%     P_{\theta,\mathbf{Z}=\mathbf{z}}(A_0)
%     P_{\theta,\mathbf{Z}=\mathbf{z}}(A_1)$.
%     Следовательно: 
%     $$P_{\theta,\mathbf{Z}=\mathbf{z}}(A_0 \cup A_1) 
%     = P_{\theta,\mathbf{Z}=\mathbf{z}}(A_0) +
%     P_{\theta,\mathbf{Z}=\mathbf{z}}(A_1) - 
%     P_{\theta,\mathbf{Z}=\mathbf{z}}(A_0)
%     P_{\theta,\mathbf{Z}=\mathbf{z}}(A_1) 
%     \geq \gamma + \gamma - \gamma^2 = \alpha
%     $$
%     Тогда:
%     $$
%     E_{\theta}(\varphi^{\text{Subsamples}}) =
%     P_{\theta}(\varphi^{\text{Subsamples}}=1)=
%     P_{\theta}(A_0 \cup A_1)=
%     $$
%     $$
%     =\sum_{\mathbf{z}} P_{\theta,\mathbf{Z}=\mathbf{z}}(A_0 \cup A_1)P_{\theta}(\mathbf{Z}=\mathbf{z}) \geq
%     \sum_{\mathbf{z}} \alpha P_{\theta}(\mathbf{Z}=\mathbf{z}) = \alpha
%     $$
    
% \end{proof}
